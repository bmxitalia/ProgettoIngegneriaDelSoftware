\section{Introduzione} \label{intro}

    \subsection{Scopo del documento}

        Questo documento è rivolto all’utente utilizzatore di \ProjectName{} e ha lo scopo di illustrare gli aspetti di base del prodotto e le configurazioni modificabili dall'utente.

    \subsection{Scopo del prodotto}

        \ScopoProdotto{}


    \subsection{Informazioni utili}
		
		\ProjectName{} può essere visto come una pipeline di eventi che:
		\begin{itemize}
			\item date le tracce di azioni intraprese su un applicazione;
			\item queste vengono analizzate, creando così delle \textit{metriche};
			\item le quali servono per analizzare il comportamente dell'applicazione, tramite la creazione di \textit{baseline};
			\item queste baseline vengono poi utilizzate per controllare che il comportamento dell'applicazione di standard;
			\item se ciò non fosse, vengono riscontrati degli \textit{alert};
			\item successivamente vengono lanciare delle \textit{azioni di rimedio}.
		\end{itemize}		

		Per questo motivo è stata scelta una struttura del documento che permetta da prima di avere familiarità con le configurazioni degli elementi sopra citati, e successivamente spieghi come svilupparle al meglio.

		Al fine di illustrare i concetti con maggior chiarezza, nell'appendice \ref{glossario} è presente un glossario
		con i termini che \GroupName{} ritiene necessitino di una definizione. L'identificazione di questi termini
		viene fatta marcando la prima occorrenza di questi con una \glossaryItem{} a pedice e il testo sarà in corsivo.
		
		\subsection{Reperimento del codice} \label{reperimentocodice}

    	Il codice sorgente di \ProjectName{} è reperibile all'indirizzo \url{https://bit.ly/2IWk45B}.
