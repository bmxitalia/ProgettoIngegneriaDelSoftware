\subsection{Alters}

	\subsubsection{Configurazione di un alert}

	
	Di seguito viene riportata un configurazione di esempio per un alert, con la successiva spiegazione dei vari campi che l'utente può configurare:

	\begin{lstlisting}[style=json]

		{
			"name": "http-duration-warning",
			"evaluate_when": "all_match",
			"conditions": [ /* ... */ ],
			"verification": {
				"on": "immediately"
			},
			"actions": [
				"VK1d4GIBNvwm-_-b4nKr",
				"wa1g4GIBNvwm-_-bK34k",
				"wq1g4GIBNvwm-_-bL35Z"
			]
		}
	\end{lstlisting}

	La configurazione di cui sopra va inserita nell'indice di configurazione alter che di default è: \textit{openapm-config-alerts} \\
	I campi configurabili sono quindi:

	\begin{itemize}
                \item \textbf{name:} specifica il nome dell'alert, a puro scopo informativo;
                \item \textbf{evaluate\_when:} specifica un identificatore che indica la strategia da utilizzare per determinare lo stato globale dell'alert a partire dallo stato delle singole condizioni;
                \item \textbf{conditions:} specifica una lista di singole condizioni che verificano questo alert;
                \item \textbf{verification:} specifica un identificatore per la strategia da utilzzare;
                \item \textbf{actions:} specifica un elenco di ID corrispondenti a delle azioni di rimedio.
	\end{itemize}

	\newpage

	\myparagraph{Il campo `conditions'}

	Espandiamo ora la clausola `conditions' per illustrarne composizione e campi configurabili:

	\begin{lstlisting}[style=json]

		{
			"metrics": [
				[
					{
					"operator": "=",
					"operands": [
						"name",
						"'http-duration-by-instance'"
						]
					},
					{
					"operator": "=",
					"operands": [
						"group",
						"'/'"
						]
					}
				],
				[
					{
						"operator": "=",
						"operands": [
							"name",
							"'http-duration-by-instance'"
							]
					},
					{
						"operator": "=",
						"operands": [
							"group",
							"'/vets.html'"
							]
					}
				]
			],
			"baseline": { /* ... */ },
			"calculation": {
				"operator": "+",
				"operands": [
					"value"
					]
			},
			"evaluation": {
				"operator": ">",
				"operands": [
					"value",
					"baseline_value"
					]
			}
		}
	\end{lstlisting}

	I campi configurabili sono quindi:

	\begin{itemize}
                \item \textbf{metrics:} specifica un elenco di metriche da considerare per calcolare lo stato della condizione. Qui ogni metrica è rappresentata da un elenco di filtri, in AND logico tra loro, per determinare quali sono valide. \\
					(Es. nel caso di cui sopra si considerano due metriche, entrambe con nome `http-duration-by-instance', appartenenti però a due gruppu diversi, rispettivamente `/' e `/vets.html');
                \item \textbf{baseline (opzionale):} specifica una baseline da utilizzare per il calcolo della condizione;
                \item \textbf{calculation:} specifica un operatore che, dato delle metriche, ricavi un valore numerico da utilizzare per il confronto;
                \item \textbf{evaluation:} specifica un operatore che, preso il valore numerico calcolato in precedenza (`value') e opzionalmente un valore della baseline (`baseline\_value'), determini se la condizione è vera o falsa.
	\end{itemize}

	\newpage

	\mysubparagraph{Il campo `baselines'}

	Espandiamo ora la clausola opzionale `baselines' per illustrarne composizione e campi configurabili:

	\begin{lstlisting}[style=json]

		{
			"index": "openapm-baselines",
			"filters": [
				{
					"operator": "=",
					"operands": [
						"name",
						"'http-duration-by-url'"
						]
				},
				{
					"operator": "=",
					"operands": [
						"group",
						"'/'"
						]
				}
			],
			"strategy": "daily",
			"calculation": {
				"operator": "+",
				"operands": [
					"mean",
					"deviation"
					]
			}
		}
	\end{lstlisting}

	I campi configurabili sono quindi:

	\begin{itemize}
                \item \textbf{index:} specifica un indice da cui prelevare le baseline;
                \item \textbf{filters:} specifica un elenco di filtri per determinare quali baseline ottenere;
                \item \textbf{strategy:} specifica una strategia per indicare il periodo di interesse delle baseline;
                \item \textbf{calculation:} specifica un operatore per determinare un valore numerico da una baseline precedentemente ottenuta.
	\end{itemize}
