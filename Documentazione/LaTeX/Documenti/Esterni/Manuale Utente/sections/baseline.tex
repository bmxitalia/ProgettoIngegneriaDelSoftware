\subsection{Baselines}

	\subsubsection{Configurazione di una baseline}

	Di seguito viene riportata un configurazione di esempio per una baseline, con la successiva spiegazione dei vari campi che l'utente può configurare:

	\begin{lstlisting}[style=json]

		{
			"name": "http-duration-by-url",
			"metrics_index": "#'openapm-metrics-'yyyy.MM.dd",
			"baselines_index": "openapm-baselines",
			"strategy": "daily",
			"filters": [
				{
				"operator": "=",
				"operands": [
					"name",
					"'http-duration-by-url'"
					]
				}
			],
			"aggregation": {
				"operands": [
				"group"
				]
			},
			"calculation_field": "value"
		}
	\end{lstlisting}

	La configurazione di cui sopra va inserita nell'indice di configurazione baseline che di default è: \textit{openapm-config-baselines} \\
	I campi configurabili sono quindi:

	\begin{itemize}
                \item \textbf{name:} specifica il nome per la baseline generata, rappresenterà il valore del campo `baseline' della metrica;
                \item \textbf{metrics\_index:} specifica l'indice o la tabella da cui prelevare le metriche;
                \item \textbf{baselines\_index:} specifica l'indice o la tabella in cui salvare le baseline;
                \item \textbf{strategy:} specifica un identificatore per la strategia da utilzzare;
                \item \textbf{filters:} specifica una lista di operatori filtro per selezionare le metriche da utilizzare per il calcolo;
                \item \textbf{aggregation:} specifica un operatore che, dato un unico gruppo di metriche, le divida in tanti sottogruppi;
                \item \textbf{calculation\_field:} specifica il campo con il valore numerico da utilizzare per il calcolo della baseline.
	\end{itemize}

