\subsection{Metriche}

	\subsubsection{Configurazione di una metrica}

	Di seguito viene riportata un configurazione di esempio per una metrica, con la successiva spiegazione dei vari campi che l'utente può configurare:

	\begin{lstlisting}[style=json]

		{
			"name": "http-duration-by-url",
			"cron": "0 * * * * *",
			"duration": 60,
			"traces_index": "stagemonitor-spans-*",
			"metrics_index": "#'openapm-metrics-'yyyy.MM.dd",
			"filters": [
				{
					"operator": "=",
					"operands": [
						"type",
						"'http'"
					]
				}
			],
			"aggregation": {
				"operands": [
				"http.url"
				]
			},
			"calculation": {
				"operator": "average",
				"operands": [
					"duration_ms"
				]
			}
		}
	\end{lstlisting}

	La configurazione di cui sopra va inserita nell'indice di configurazione metriche che di default è: \textit{openapm-config-metrics} \\
	I campi configurabili sono quindi:

	\begin{itemize}
                \item \textbf{name:} specifica il nome per la metrica generata, rappresenterà il valore del campo `name' della metrica;
                \item \textbf{cron:} specifica un'espressione cron che determini quando il calcolo della metrica deve essere eseguito;
                \item \textbf{duration:} specifica un valore, in secondi, che determini l'intervallo in cui prelevare le trace per il calcolo la metrica (Es. 60 secondi);
                \item \textbf{traces\_index:} specifica l'indice o la tabella da cui prelevare le trace;
                \item \textbf{metrics\_index:} specifica l'indice o la tabella in cui salvare le metriche;
                \item \textbf{filters:} specifica una lista di operatori filtro per selezionare le trace valide da utilizzare per il calcolo;
                \item \textbf{aggregation:} specifica un operatore che, dato un unico gruppo di trace, le divida in tanti sottogruppi;
                \item \textbf{calculation:} specifica un operatore che, dato un gruppo di metriche, ne estragga un valore numerico rappresentante il valore finale della metrica per quel gruppo di trace.
	\end{itemize}
