\section{Descrizione generale} \label{descrizione generale}

    \subsection{Obiettivi del prodotto}

        Il progetto ha come obiettivo la creazione di un applicativo che interpreti periodicamente dati raccolti da un
        \glossaryItem{Agent}. L'applicativo dovrà essere in grado di:
        \begin{itemize}
            \item recuperare i dati periodicamente da \glossaryItem{ElasticSearch};
        	\item generare metriche dai dati;
        	\item generare baseline basate sulle metriche del punto precedente;
        	\item inviare notifiche allo scatenarsi di critical event.
		\end{itemize}

    \subsection{Funzioni del prodotto}

        L'applicativo fornirà le seguenti funzioni:

        \begin{itemize}

            \item creazione ed avvio periodico di una procedura automatica che autonomamente accederà al resto delle
            funzionalità offerte dall'applicativo;
            \item raccolta di traces da un indice di ElasticSearch a intervalli regolari, ad esempio un minuto, ed
            elaborazione di tali dati per creare delle metriche utili alle altre funzionalità;
            \item calcolo automatico di baseline su ogni metrica calcolata e utilizzo delle stesse per determinare, ad ogni
            intervallo di raccolta traces, se sono state superate soglie configurabili;
            \item notifica del superamento di determinate soglie all'utente utilizzatore dell'applicativo, tramite
            invio di messaggi di posta elettronica;
            \item esecuzione di procedure automatiche al superamento di determinate soglie configurabili;
            \item inserimento di informazioni utili in un indice dedicato, al verificarsi di un critical event.

        \end{itemize}
        
   	\subsection{Funzionalità necessarie da non implementare} \label{funzioni non richieste}
   	
   		Nello sviluppo di un applicativo di questo tipo è necessario un \textbf{meccanismo di autenticazione} e
   		un'\textbf{interfaccia grafica} per l'utilizzo. Il gruppo non si occuperà dello sviluppo di queste
   		componenti perché la \glossaryItem{Proponente} possiede già delle soluzioni che
   		ritiene valide.

    \subsection{Ambiente di esecuzione}

    	La Proponente \Proponente{} fornisce al gruppo \GroupName{} il seguente ambiente di esecuzione:

    		\begin{itemize}
    			\item un server contenente un applicativo da monitorare che genera traces;
    			\item un server dove sono installati ElasticSearch e \glossaryItem{Kibana}, dove devono essere raccolte e visualizzate le traces. In 							questo server deve essere installata l'applicazione da realizzare.
    		\end{itemize}

    \subsection{Vincoli} \label{vincoli}

        Per il corretto funzionamento dell'applicativo è necessario che nel server ElasticSearch, con cui comunica
        il sistema, siano presenti le seguenti configurazioni:

            \begin{itemize}
                \item schedulazione della procedura batch;
                \item parametri di filtraggio delle traces per calcolo metriche e aggiornamento baseline;
                \item parametri di raggruppamento delle traces per calcolo metriche e aggiornamento baseline;
                \item \glossaryItem{policy} utilizzata per il controllo di critical event;
                \item template della mail da utilizzare per l'invio di mail allo scatenarsi di critical event;
                \item configurazione per l'invio delle mail allo scatenarsi di critical event;
                \item indicazioni sulla procedura automatica da eseguire allo scatenarsi di critical event che
                la richiedano.
            \end{itemize}

        L'applicazione può essere eseguita nel sistema operativo Debian con installato Java 8.0.
