\subsection{Requisiti negoziati con la Proponente} \label{negoziazione}
	In seguito all'incontro avvenuto con la Proponente in data 2018-03-02 è stato deciso di rimuovere i casi d'uso relativi al Machine 				Learning e, di conseguenza, anche i relativi requisiti.\\
	La negoziazione di tali requisiti opzionali è derivata dal verificarsi di alcuni rischi, analizzati in \vPianoDiProgetto, in particolare 			quello relativo alla disponibilità temporale.\\
	Dato che la negoziazione dei requisiti può essere fatta solo in maniera migliorabile, il team ha deciso di sfruttare il tempo guadagnato 		per:
	\begin{itemize}
		\item rendere i requisiti obbligatori più solidi;
		\item migliorare l'efficacia dell'intero progetto.
	\end{itemize}
            
\subsection{Requisiti funzionali} \label{requisiti fun}
\begin{center}
    \begin{longtable}{ | >{\centering\arraybackslash}m{2.5cm} | >{\centering\arraybackslash}m{8cm} | >{\centering\arraybackslash}m{2.5cm} | }
        
        \hline
        \textbf{Id Requisito} & \textbf{Descrizione} & \textbf{Fonti} \\ \hline
        \endhead
        RFO1 & L'amministratore di sistema deve poter avviare una procedura Batch secondo configurazioni temporali prestabilite & Interno UC2 \\ \hline
        RFO1.1 & Per poter avviare una procedura Batch, l'amministratore di sistema, deve accedere al sistema tramite il modulo di 						autenticazione fornito da \Proponente{} & Interno UC2 \\ \hline
        RFO1.2 & Dopo aver eseguito l'accesso correttamente, l'amministratore di sistema deve poter avviare una procedura Batch & Interno UC2 \\ \hline
        RFO2 & La procedura batch deve essere in grado di generare una metrica & Capitolato UC3\\ \hline
        RFO2.1 & La procedura batch deve essere in grado di leggere trace da un indice ElasticSearch contenente le trace & Capitolato UC3.1\\ \hline
        RFO2.2 & La procedura batch deve poter filtrare le trace prima di fare dei raggruppamenti su di esse & Capitolato UC3.2\\ \hline
        RFO2.2.1 & La procedura batch deve leggere la modalità di filtraggio delle trace da un indice su ElasticSearch & Capitolato UC3.2\\ \hline
        RFO2.2.2 & La procedura batch deve leggere il valore per la modalità filtraggio scelta da un indice su ElasticSearch & Capitolato UC3.2\\ \hline
        RFO2.2.3 & La procedura batch deve filtrare le trace in base alla configurazione di filtraggio scelta & Capitolato UC3.2\\ \hline
        RFO2.3 & La procedura batch deve essere in grado di raggruppare delle trace in base a dei parametri configurabili & Interno UC3.3\\ \hline
        RFO2.3.1 & La procedura batch deve essere in grado di leggere la modalità di raggruppamento delle trace da un indice ElasticSearch & Interno UC3.3 \\ \hline
        RFO2.3.1.1 & La procedura batch deve poter raggruppare trace secondo un attributo specificato & Capitolato UC3.3\\ \hline
        RFO2.3.1.2 & La procedura batch deve poter raggruppare trace per host & Interno UC3.3\\ \hline
        RFO2.3.1.3 & La procedura batch deve poter raggruppare trace per path di una richiesta HTTP & Interno UC3.3\\ \hline
        RFO2.3.1.4 & La procedura batch deve poter raggruppare trace per parte di path di una richiesta HTTP & Interno UC3.3\\ \hline
        RFO2.3.1.5 & La procedura batch deve poter raggruppare trace per tipologia di query SQL & Interno UC3.3\\ \hline
        RFO2.3.1.6 & La procedura batch deve poter raggruppare trace per tempo di esecuzione di una determinata query & Interno UC3.3\\ \hline
        RFO2.3.1.7 & La procedura batch deve poter raggruppare trace per tempo di risposta di una richiesta HTTP & Interno UC3.3\\ \hline
        RFO2.3.1.8 & La procedura batch deve poter raggruppare trace per tipologia di richiesta (http o jdbc) & Interno UC3.3\\ \hline
        RFO2.3.1.9 & La procedura batch deve poter raggruppare trace per IP di provenienza della richiesta & Interno UC3.3\\ \hline
        RFO2.3.1.10 & La procedura batch deve poter raggruppare trace per tipologia di browser con cui si è fatta la richiesta & Interno UC3.3\\ \hline
        RFO2.3.1.11 & La procedura batch deve poter raggruppare trace per intervalli di tempo in cui sono avvenute le richieste & Interno UC3.3\\ \hline
        RFO2.3.1.11.1 & La procedura batch deve poter raggruppare trace avvenute in una certa ora del giorno & Interno UC3.3\\ \hline
        RFO2.3.1.11.2 & La procedura batch deve poter raggruppare trace avvenute un certo giorno della settimana & Interno UC3.3\\ \hline
        RFO2.3.1.11.3 & La procedura batch deve poter raggruppare trace avvenute in un certo mese dell'anno & Interno UC3.3\\ \hline
        RFO2.3.2 & La procedura batch deve poter scegliere il valore per il parametro di raggruppamento, prelevandolo da un indice ElasticSearch & Interno UC3.3\\ \hline
        RFO2.3.2.1 & La procedura batch deve poter prelevare un valore per un attributo specificato & Interno UC3.3\\ \hline
        RFO2.3.2.2 & La procedura batch deve poter prelevare un valore per un parametro host & Interno UC3.3\\ \hline
        RFO2.3.2.3 & La procedura batch deve poter prelevare un valore per un parametro path & Interno UC3.3\\ \hline
        RFO2.3.2.4 & La procedura batch deve poter prelevare un valore per una parte di path & Interno UC3.3\\ \hline
        RFO2.3.2.5 & La procedura batch deve poter prelevare un valore per una query SQL & Interno UC3.3\\ \hline
        RFO2.3.2.6 & La procedura batch deve poter prelevare un valore per il tempo di esecuzione di una query SQL & Interno UC3.3\\ \hline
        RFO2.3.2.7 & La procedura batch deve poter prelevare un valore per il tempo di risposta di una richiesta HTTP & Interno UC3.3\\ \hline
        RFO2.3.2.8 & La procedura batch deve poter prelevare un valore per il parametro tipologia di richiesta & Interno UC3.3\\ \hline
        RFO2.3.2.9 & La procedura batch deve poter prelevare un valore per il parametro tipologia di browser & Interno UC3.3\\ \hline
        RFO2.3.2.10 & La procedura batch deve poter prelevare un valore per un intervallo di tempo in cui sono avvenute delle richieste & Interno UC3.3\\ \hline
        RFO2.3.2.10.1 & La procedura batch deve poter prelevare un valore per il parametro di intervallo di tempo \textbf{ora} & Interno UC3.3\\ \hline
        RFO2.3.2.10.2 & La procedura batch deve poter prelevare un valore per il parametro di intervallo di tempo \textbf{mese} & Interno UC3.3\\ \hline
        RFO2.3.2.10.3 & La procedura batch deve poter prelevare un valore per il parametro di intervallo di tempo \textbf{anno} & Interno UC3.3\\ \hline
        RFO2.3.3 & La procedura batch deve poter raggruppare le trace in base alla modalità di raggruppamento scelta e il valore del parametro scelto per configurarla & Interno UC3.3\\ \hline
       	RFO2.4 & La procedura batch deve essere in grado di calcolare una metrica & Capitolato 	UC3.4\\ \hline
       	RFO2.4.1 & La procedura batch deve leggere la tipologia di metrica da calcolare da un indice ElasticSearch &  Interno UC3.4\\ \hline
       	RFO2.4.1.1 & La procedura può scegliere di calcolare la metrica - Numero medio di errori & Capitolato UC3.4\\ \hline
       	RFO2.4.1.2 & La procedura può scegliere di calcolare la metrica - Tempo medio di risposta & Capitolato UC3.4\\ \hline
       	RFO2.4.1.3 & La procedura può scegliere di calcolare la metrica - Tempo massimo di risposta & Capitolato UC3.4\\ \hline
       	RFO2.4.1.4 & La procedura può scegliere di calcolare la metrica - Tempo minimo di risposta & Capitolato UC3.4\\ \hline
       	RFO2.4.1.5 & La procedura può scegliere di calcolare la metrica - Numero medio di chiamate & Capitolato UC3.4\\ \hline
       	RFO2.4.2 & La procedura batch deve poter scegliere la \glossaryItem{granularità} di tempo per il calcolo della metrica & Capitolato UC3.4\\ \hline
       	RFO2.4.2.1 & La procedura batch deve poter scegliere una granularità di un minuto & Capitolato UC3.4\\ \hline
       	RFO2.4.2.2 & La procedura batch deve poter scegliere una granularità di un'ora & Capitolato UC3.4\\ \hline
       	RFO2.4.2.3 & La procedura batch deve poter poter calcolare il valore per una metrica in tempo reale & Capitolato UC3.4\\ \hline
       	RFO2.4.3 & La procedura batch deve poter calcolare metriche basandosi sullo storico delle metriche & Capitolato UC3.4\\ \hline
       	RFO2.4.4 & La procedura batch deve poter calcolare la metrica in base alla tipologia scelta & Interno UC3.4\\ \hline
       	RFO2.4.5 & La procedura batch deve generare un file JSON contenente la metrica calcolata & Capitolato UC3.4\\ \hline
       	RFO2.5 & La procedura batch deve poter salvare la metrica calcolata su un indice ElasticSearch & Capitolato UC3.5\\ \hline
       	RFO3 & L'inserimento di una metrica deve scatenare la creazione di una baseline basata su tale metrica, da parte della procedura batch & Capitolato UC4\\ \hline
       	RFO3.1 & L'inserimento di una metrica deve scatenare l'aggiornamento di una baseline per tale metrica, nel caso in cui la baseline esista già & Interno UC4\\ \hline
       	RFO3.1.1 & Per la creazione di una baseline, la procedura batch deve poter scegliere una configurazione temporale, prelevata da un indice ElasticSearch & Capitolato UC4.1\\ \hline
       	RFO3.1.1.1 & Per la costruzione di una baseline, la procedura batch può scegliere una base oraria con modello giornaliero & Capitolato UC4.1\\ \hline
       	RFO3.1.1.2 & Per la costruzione di una baseline, la procedura batch può scegliere una base oraria con modello settimanale & Capitolato UC4.1 \\ \hline
       	RFO3.1.1.3 & Per la costruzione di una baseline, la procedura batch può scegliere una base oraria con modello mensile & Capitolato UC4.1 \\ \hline
       	RFO3.1.2 & Per la costruzione di una baseline, la procedura batch deve poter leggere le metriche coinvolte nel calcolo da un indice ElasticSearch & Capitolato UC4.1 \\ \hline
       	RFO3.1.3 & La procedura batch deve essere in grado di calcolare una baseline & Capitolato UC4.1\\ \hline
       	RFO3.1.3.1 & La procedura batch deve poter calcolare la media delle metriche coinvolte nel calcolo & Capitolato UC4.1\\ \hline
       	RFO3.1.3.2 & La procedura batch deve poter calcolare la deviazione standard delle metriche coinvolte nel calcolo & Capitolato UC4.1\\ \hline
       	RFO3.1.3.3 & La procedura batch deve poter generare la baseline in base ai calcoli effettuati & Capitolato UC4.1\\ \hline
       	RFO3.1.3.4 & La procedura batch deve poter generare un file JSON contenente la baseline calcolata & Interno UC4.1\\ \hline
        RFO3.1.4 & La procedura batch deve generare baseline in base alla configurazione temporale decisa & Capitolato UC4.1\\ \hline
        RFO3.1.5 & La procedura batch deve poter salvare la baseline calcolata in un indice ElasticSearch & Interno UC4.2\\ \hline
       	RFO4 & L'inserimento di una nuova metrica deve scatenare un controllo critical event da parte della procedura batch & Capitolato UC5\\ \hline
       	RFO4.1 & La procedura batch deve poter configurare una policy, che può avere anche più condizioni associate, leggendo dati da un indice ElasticSearch & Capitolato UC5.1\\ \hline
       	RFO4.1.1 & La procedura batch deve selezionare una tipologia di soglia per la policy da un indice ElasticSearch & Interno UC5.1\\ \hline
       	RFO4.1.1.1 & La procedura batch può selezionare una soglia statica & Capitolato UC5.1\\ \hline
       	RFO4.1.1.2 & La procedura batch può selezionare una soglia dinamica, ovvero una baseline & Capitolato UC5.1\\ \hline
       	RFO4.1.1.3 & La procedura batch può selezionare una baseline con deviazione standard come soglia & Capitolato UC5.1\\ \hline
       	RFO4.1.2 & La procedura batch deve poter leggere un valore per la soglia scelta da un indice ElasticSearch & Interno UC5.1\\ \hline
       	RFO4.1.2.1 & La procedura batch deve poter leggere un valore per una soglia statica & Interno UC5.1\\ \hline
       	RFO4.1.2.2 & La procedura batch deve poter leggere un valore per una baseline senza deviazione standard & Interno UC5.1\\ \hline
       	RFO4.1.2.3 & La procedura batch deve poter leggere un valore per una baseline con deviazione standard & Interno UC5.1\\ \hline
       	RFO4.1.3 & La procedura batch, nel caso in cui si verifichi un critical event, deve poter leggere l'azione da eseguire da un indice ElasticSearch & Interno UC5.1\\ \hline
       	RFO4.2 & La procedura batch deve poter verificare la policy, ossia se scatenare un critical event & Interno UC5.2\\ \hline
       	RFO4.2.1 & La procedura batch deve leggere il valore attuale della metrica inserita & Interno UC5.2\\ \hline
       	RFO4.2.2 & La procedura batch deve verificare se il valore della metrica è in linea con la soglia selezionata & Interno UC5.2\\ \hline
       	RFO4.3 & La procedura batch deve lanciare un critical event nel caso in cui la soglia viene superata & Capitolato UC5.2\\ \hline
       	RFO4.3.1 & La procedura batch può lanciare un critical event immediatamente & Capitolato UC5.2\\ \hline
       	RFO4.3.2 & La procedura batch può lanciare un critical event dopo N minuti che si è verificata la criticità & Capitolato UC5.2\\ \hline
       	RFO4.3.3 & La procedura batch può lanciare un critical event alla terminazione della criticità & Capitolato UC5.2\\ \hline
       	RFO4.4 & La procedura batch, dopo aver lanciato un critical event, può eseguire un'azione & Capitolato UC5.2\\ \hline
       	RFO4.4.1 & La procedura batch può inviare una e-mail di notifica del critical event & Capitolato UC5.2\\ \hline
       	RFO4.4.2 & La procedura batch può eseguire una procedura automatica & Capitolato UC5.2\\ \hline
       	RFO4.4.3 & La procedura batch può salvare il critical event & Capitolato UC5.2\\ \hline
       	RFD5 & Allo scattare di un critical event, la procedura batch, deve poter inviare una e-mail di notifica & Capitolato UC6\\ \hline
       	RFD5.1 & La procedura batch deve poter prelevare l'indirizzo e-mail del destinatario da un indice ElasticSearch & Interno UC6.1 \\ \hline
       	RFD5.2 & La mail deve essere creata tramite template JTwig & Capitolato UC6.2\\ \hline
       	RFD5.3 & La procedura batch deve configurare la e-mail leggendo la configurazione da un indice ElasticSearch & Capitolato UC6.2\\ \hline
       	RFD5.3.1 & La procedura batch deve poter leggere le configurazioni di invio della e-mail da un indice ElasticSearch & Interno UC6.2\\ \hline
       	RFD5.3.1.1 & La procedura batch deve leggere il server SMTP & Interno UC6.2\\ \hline
       	RFD5.3.1.2 & La procedura batch deve leggere il numero di porta & Interno UC6.2\\ \hline
       	RFD5.3.1.3 & La procedura batch deve leggere l'oggetto della e-mail & Interno UC6.2\\ \hline
       	RFD5.3.1.4 & La procedura batch deve leggere lo username dell'account di posta elettronica & Interno UC6.2\\ 	\hline
       	RFD5.3.1.5 & La procedura batch deve leggere la password dell'account di posta elettronica & Interno UC6.2\\ \hline
       	RFD5.4 & La procedura batch deve leggere il testo della e-mail (può essere codice HTML) da un indice ElasticSearch & Interno UC6.2\\ \hline
       	RFD5.5 & La procedura batch deve collegarsi al server di invio della e-mail & Interno UC6.3.1\\ \hline
       	
       	RFD5.6 & La procedura batch può inviare la e-mail al destinatario scelto e con le configurazioni impostate & Interno UC6.3 UC6.3.2\\ \hline
       	RFD6 & Allo scattare di un critical event, la procedura batch, deve poter memorizzare il critical event & Capitolato UC8\\ \hline
       	RFD6.1 & La procedura batch deve poter prelevare l'indice ElasticSearch di salvataggio da un indice ElasticSearch & Interno UC8\\ \hline
       	RFD6.2 & La procedura batch deve inserire il critical event in un file JSON & Interno UC8\\ \hline
       	RFD6.3 & La procedura batch può memorizzare il critical event sull'indice ElasticSearch prelevato & Interno UC8\\ \hline
       	RFD7 & Allo scattare di un critical event, la procedura batch, deve poter eseguire una procedura automatica & Capitolato UC9\\ \hline
       	RFD7.1 & La procedura batch deve poter prelevare la procedura da eseguire da un indice ElasticSearch & Interno UC9\\ \hline
       	RFD7.2 & La procedura batch può eseguire la procedura prelevata tramite Script \glossaryItem{Bash} & Capitolato UC9\\ \hline
       	
       	RFF8 & L'amministratore di sistema deve essere in grado di configurare la schedulazione della procedura batch da eseguire & Capitolato 		UC1\\ \hline
       	RFF8.1 & L'amministratore di sistema deve poter leggere la configurazione temporale della procedura batch da un indice ElasticSearch & 					Capitolato UC1.1\\ \hline
       	RFF8.2 & L'amministratore di sistema deve poter configurare la procedura batch con il parametro letto & Capitolato UC1.1\\ \hline
       	RFF8.3 & L'amministratore di sistema deve poter memorizzare in un indice ElasticSearch la nuova configurazione della procedura batch & 		Capitolato UC1.2\\ \hline
       	RFF9 & In caso di errore di connessione al server di invio della e-mail, la procedura batch deve poter memorizzare il messaggio per un 
       	invio futuro & Interno UC7 \\ \hline
       	RFF9.1 & La procedura batch deve prelevare da un indice ElasticSearch la posizione in cui memorizzare il messaggio & Interno UC7 \\ \hline
       	RFF9.2 & La procedura batch deve poter memorizzare il messaggio sull'indice ElasticSearch prelevato & Interno UC7 \\ \hline
       	\caption[Requisiti funzionali]{Tabella dei requisiti funzionali}
	\end{longtable}
	
\end{center}