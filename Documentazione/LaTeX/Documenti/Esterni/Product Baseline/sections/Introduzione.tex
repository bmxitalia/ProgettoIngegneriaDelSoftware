\section{Introduzione} \label{intro}

    \subsection{Scopo del documento}

        Il seguente documento ha lo scopo di illustrare e descrivere l'architettura progettata per l'applicazione
        realizzata durante il \glossaryItem{progetto} OpenAPM.

        Descrivendo l'architettura generale e andando in dettaglio spiegando classi e \glossaryItem{design
        pattern} utilizzati, il lettore potrà avere una visione d'insieme delle componenti dell'applicazione, avendo
        quindi una maggiore consapevolezza sulle scelte adottate dal \glossaryItem{team}.

    \subsection{Scopo del prodotto}

        \ScopoProdotto{}

    \subsection{Premessa}

        Data la complessità dell'architettura, alcuni diagrammi possono richiedere uno zoom per essere visionati correttamente.

    \subsection{Riferimenti}

        \subsubsection{Riferimenti normativi}

            \begin{itemize}
    	        \item
    	            \textbf{Norme di progetto}: \vNormeDiProgetto{};
    	    \end{itemize}

        \subsubsection{Riferimenti informativi}

            \item
                \textbf{Spring Batch}\\
                \url{https://projects.spring.io/spring-batch/}\\
                (ultima consultazione effettuata in data 2018-06-05);

            \item
                \textbf{Spring e-mail}\\
                \url{http://www.baeldung.com/spring-email}\\
                (ultima consultazione effettuata in data 2018-06-05);
            \item
                \textbf{ElasticSearch}\\
                \url{https://www.elastic.co/products/elasticsearch} \\
                (ultima consultazione effettuata in data 2018-06-05);

            \item
                \textbf{Spring} \\
                \url{https://spring.io/} \\
                (ultima consultazione effettuata in data 2018-06-05).

    \subsection{Glossario}

        \DescrizioneGlossario{}
