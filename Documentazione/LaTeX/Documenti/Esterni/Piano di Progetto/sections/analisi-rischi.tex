\section{Analisi dei rischi} \label{analisi dei rischi}

Al fine di migliorare la qualità del progetto viene presentata di seguito un'analisi realistica dei
rischi che potrebbero insorgere nel corso dello sviluppo.

\begin{small}
    \begin{center}
        \renewcommand{\arraystretch}{1.25}
        \rowcolors{2}{gray!25}{gray!5}
        \begin{longtable}{ >{\centering\arraybackslash}m{2.5cm} m{4cm} m{4cm} >{\centering\arraybackslash}m{2.5cm} }
            \rowcolor{gray!50}
            Nome & Descrizione & Rilevamento & Grado di rischio \\
            \hline
            \endhead
            
            \textbf{Tecnologie da usare}
            &
            Il tempo richiesto per l'apprendimento delle tecnologie da parte del gruppo
            potrebbe causare ritardi nello sviluppo.
            &
            Ogni membro comunicherà al \Responsabile{} lo stato della propria preparazione.
            &
            Occorrenza: \textbf{Alta} Pericolosità: \textbf{Alta}
            \\
            Piano di contingenza: &
            \multicolumn{3}{ m{11.35cm} } {
                Il carico di lavoro verrà ridistribuito in caso di lacune da parte di alcuni membri.
            } \\
            \hline
            
            \textbf{Scarsa esperienza}
            &
            Nessun membro del gruppo ha mai lavorato a un progetto così impegnativo, 
            ciò potrebbe tradursi in ritardi dovuti all'inesperienza.
            &
            Ogni membro comunicherà al \Responsabile{} eventuali difficoltà.
            &
            Occorrenza: \textbf{Alta} Pericolosità: \textbf{Alta}
            \\
            Piano di contingenza: &
            \multicolumn{3}{ m{11.35cm} } {
                I compiti di maggior difficoltà verranno affidati ai membri con più conoscenze.
            } \\
            \hline
            
            \textbf{Stime dei costi}
            &
            I membri non hanno esperienza nella pianificazione del progetto,
            questo può portare a stime errate dei costi.
            &
            Ogni membro comunicherà al \Responsabile{} stime errate del proprio lavoro.
            &
            Occorrenza: \textbf{Medio-alta} Pericolosità: \textbf{Media}
            \\
            Piano di contingenza: &
            \multicolumn{3}{ m{11.35cm} } {
                Il \Responsabile{} provvederà a ridistribuire il lavoro in caso di stime errate.
            } \\
            \hline
            
            \textbf{Disponibilità temporali}
            &
            Tutti i membri di \GroupName{} sono studenti e parte di essi è anche lavoratore.
            A causa di impegni il tempo da dedicare al progetto potrebbe essere limitato.
            &
            Ogni membro comunicherà al \Responsabile{} i propri impegni.
            &
            Occorrenza: \textbf{Media} Pericolosità: \textbf{Medio-alta}
            \\
            Piano di contingenza: &
            \multicolumn{3}{ m{11.35cm} } {
                Il carico di lavoro verrà distribuito in base agli impegni dei membri.
            } \\
            \hline
            
            \textbf{Contrasti nel gruppo}
            &
            Nessun membro si è mai confrontato con un gruppo così ampio di collaboratori, 
            inoltre nessun membro conosceva gli altri prima della formazione del gruppo.
            Questo potrebbe portare a contrasti e tensioni.
            &
            Il \Responsabile{} dovrà monitorare comunicazioni e coordinazioni tra i membri.
            &
            Occorrenza: \textbf{Medio-bassa} Pericolosità: \textbf{Medio-alta}
            \\
            Piano di contingenza: &
            \multicolumn{3}{ m{11.35cm} } {
                Il \Responsabile{} agirà da mediatore nei momenti di tensione.
            } \\
            \hline
            
            \textbf{Analisi dei Requisiti errata}
            &
            Data la scarsa esperienza di \GroupName{} l'Analisi dei Requisiti potrebbe risultare 
            errata o incompleta.
            &
            In caso di errori si cercherà un riscontro con \Proponente{}.
            &
            Occorrenza: \textbf{Bassa} Pericolosità: \textbf{Molto Alta}
            \\
            Piano di contingenza: &
            \multicolumn{3}{ m{11.35cm} } {
                Eventuali incoerenze riscontrate con \Proponente{} verranno tempestivamente corrette.
            } \\
            \hline
            
            \textbf{Modifica dei requisiti}
            &
            Nonostante i requisiti esposti inizialmente siano chiari vi è la possibilità
            che questi vengano modificati da \Proponente{}.
            &
            Ricevendo continuamente feedback da parte della \glossaryItem{Proponente}.
            &
            Occorrenza: \textbf{Molto Bassa} Pericolosità: \textbf{Alta}
            \\
            Piano di contingenza: &
            \multicolumn{3}{ m{11.35cm} } {
                In caso di cambiamenti eccessivi si cercherà un accordo con \Proponente{}.
            } \\
            \hline
            
            \textbf{Strumentazione Personale}
            &
            Ogni membro utilizza il proprio computer per lavorare al progetto,
            guasti potrebbero causare perdita di dati e ritardi o impossibilità
            di sviluppo.
            &
            Ogni membro dovrà avvisare in caso di malfunzionamento della propria attrezzatura.
            &
            Occorrenza: \textbf{Molto Bassa} Pericolosità: \textbf{Media}
            \\
            Piano di contingenza: &
            \multicolumn{3}{ m{11.35cm} } {
                In caso di perdite di dati i membri coinvolti dovranno occuparsi del ripristino.
            } \\
            \hline
            
            \textbf{Strumenti Software}
            &
            Il gruppo si affida a software di terze parti per pianificare e coordinare
            il proprio lavoro.
            &
            Non è pianificabile un metodo di rilevamento poiché dipende da fattori esterni.
            &
            Occorrenza: \textbf{Molto Bassa} Pericolosità: \textbf{Medio-bassa}
            \\
            Piano di contingenza: &
            \multicolumn{3}{ m{11.35cm} } {
                Durante la scelta degli strumenti verrà valutata l'affidabilità degli stessi.
            } \\
            \hline
            
            \textbf{Misurazioni errate}
            &
            Il gruppo effettua misurazioni su processi e prodotto, in base a metriche stabilite, durante tutto l'arco del progetto didattico. 				A causa della scarsa esperienza del team, queste misurazioni
            potrebbero avvenire in modo non accurato.
            &
            Misurazioni errate susciteranno dubbi da parte dei Verificatori durante la visione dei diagrammi a cruscotto. In caso di
            perplessità, sarà compito degli stessi informare il Responsabile di Progetto.
            &
            Occorrenza: \textbf{Medio-alta} Pericolosità: \textbf{Medio-bassa}
            \\
            Piano di contingenza: &
            \multicolumn{3}{ m{11.35cm} } {
                Il Responsabile di Progetto deciderà se le misurazioni saranno corrette o meno in base a \glossaryItem{tendenze} o 								\glossaryItem{benchmark}.
                Nel caso in cui non lo siano, affiderà ai Verificatori il compito di misurare nuovamente i processi o i moduli di prodotto
                interessati.
            } \\
            \hline
            
            \textbf{Problemi di versionamento}
            &
            Il gruppo utilizza il servizio di hosting per il versionamento \glossaryItem{GitHub}. A causa della scarsa esperienza del team, potrebbero
            essere rilevati problemi quali conflitti tra branch e merge non corretti a causa degli stessi.
            &
            Questi problemi possono essere rilevati dai Verificatori al momento della pull da un branch. Se i Verificatori si accorgono
            della mancanza di parti di codice o di documento possono verificare se questa è dovuta ad un precedente merge sul branch corrente.
            Dopo la rilevazione, i Verificatori dovranno avvisare il Responsabile di Progetto del problema.
            &
            Occorrenza: \textbf{Medio-alta} Pericolosità: \textbf{Alta}
            \\
            Piano di contingenza: &
            \multicolumn{3}{ m{11.35cm} } {
                Il Responsabile di Progetto dovrà riportare la repository in uno stato stabile. Si occuperà poi di tracciare le modifiche
                che avevano portato ai problemi per committarle correttamente.
            } \\
            \hline

            \rowcolor{white}
            \caption[Analisi dei rischi]{Tabella di analisi dei rischi}
        \end{longtable}
    \end{center}
\end{small}
