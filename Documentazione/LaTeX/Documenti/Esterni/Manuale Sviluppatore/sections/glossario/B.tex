\section*{B}

    \subsection*{Baseline}

        Nel ciclo di vita di un progetto, essa è punto d'arrivo tecnico dal quale non si retrocede. Un progetto prevede una successione
        di baseline che va gestita con processi dedicati. Identificare bene le baseline durante la pianificazione di un progetto è
        importante per garantire:
        \begin{itemize}
        	\item riproducibilità;
        	\item tracciabilità;
        	\item analisi, valutazione, confronto.
        \end{itemize}
        
        \\
        Una baseline, nel contesto del progetto didattico OpenAPM, è un punto base che viene associato a 
        delle metriche. Se le metriche si scostano troppo dalla baseline viene evidenziata una criticità 
        del sistema, il qualche deve gestire la situazione con un'azione di rimedio.
        
     \subsection*{Boilerplate}
     
     	Sezioni di codice che devono essere aggiunte in varie posizioni con minime o nulle variazioni.