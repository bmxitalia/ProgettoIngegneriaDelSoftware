\section[Configurazione ambiente di lavoro]{Configurazione dell'ambiente di lavoro}
\label{settings}

	Per modificare il comportamento del prodotto è necessario modificarne le impostazioni;
	queste sono contenute nel file \lstinline[columns=fixed]{src/main/resources/application.properties}.


	\subsection{Configurazione del server Elasticsearch}
	
		Viene qui descritto come cambiare le impostazioni del server Elasticsearch nel caso 
		si voglia modificare i parametri di default scelti da \GroupName{}.

		Per cambiare l'\eng{host} e la porta da utilizzare per comunicare con Elasticsearch, modificare la seguente chiave:
		\begin{lstlisting}
		spring.data.jest.uri=http://localhost:9200
		\end{lstlisting}
		
		Per modificare gli indici e la tipologia di record per le differenti configurazioni modificare le seguenti chiavi:
		\begin{lstlisting}[language=Python]
		# Metrics Generation
		app.metrics.config.index=openapm-config-metrics
		app.metrics.config.type=config_metrics

		# Baseline Generation
		app.baselines.config.index=openapm-config-baselines
		app.baselines.config.type=config_baselines
		app.baselines.data_points=4

		# Critical Events Checking
		app.alerts.config.index=openapm-config-alerts
		app.alerts.config.type=config_alerts
		app.alerts.evaluate.debounce=5000

		# Remediation Actions
		app.actions.config.index=openapm-config-actions
		app.actions.config.type=config_actions
		\end{lstlisting}

		La chiave \lstinline[columns=fixed]{app.alerts.evaluate.debounce} indica dopo quanto tempo dall'\eng{update} 
		dell'ultima metrica vada ricalcolato lo stato dell'\eng{alert}.

	\subsection{Configurazione del database} \label{impostazionidatabase}
	
		È possibile configurare database diversi (anche di tipologie diverse) per ogni tipologia di dato 
		(trace, metriche, baseline e allarmi scattati).\\
		Per farlo è sufficiente configurare le seguenti chiavi:\\
		\begin{lstlisting}[language=Python]
		app.traces.database     # Trace
		app.metrics.database    # Metriche
		app.baselines.database  # Baseline
		app.actions.database    # Allarmi scattati
		\end{lstlisting}

		Ogni chiave avrà una sottochiave \lstinline[columns=fixed]{driver} indicante la tipologia di database, 
		ed un insieme di altri chiavi dipendenti dal tipo di database.

		Nel caso si voglia aggiungere ad \ProjectName{} ulteriori database a quelli già disponibili,
		la §\ref{newdatabase} descrive la procedura necessaria per implementarne di nuovi.

		\subsubsection{Database disponibili}

			\myparagraph{Elasticsearch}

				\textbf{Driver}: \lstinline[columns=fixed]{elasticsearch}

				\textbf{Parametri richiesti:}
				\begin{itemize}
					\item \lstinline[columns=fixed]{host}: Host per la connessione con il server Elasticsearch;
					\item \lstinline[columns=fixed]{port}: Porta per la connessione con il server Elasticsearch;
					\item \lstinline[columns=fixed]{protocol}: Protocollo di connessione (http oppure https);
					\item \lstinline[columns=fixed]{timestamp}: Campo da utilizzare all'interno di 
							un documento per determinare quando è stato creato;
					\item \lstinline[columns=fixed]{type}: Tipologia di documenti da indicare ad Elasticsearch;
					\item \lstinline[columns=fixed]{keepalive}: Tempo in secondi prima di considerare una chiamata fallita.				
				\end{itemize}
	
				\textbf{Esempio:}
				\begin{lstlisting}
		app.traces.database.driver=elasticsearch
		app.traces.database.host=localhost
		app.traces.database.port=9200
		app.traces.database.protocol=http
		app.traces.database.timestamp=@timestamp
		app.traces.database.type=spans
		app.traces.database.keepalive=60
				\end{lstlisting}

	\subsection{Configurazione del server mail}

		Per l'invio di email viene utilizzato Spring Mail. Per una descrizione 
		più approfondita di questo framework esterno si rimanda alla sua documentazione
		presente all'indirizzo \url{https://bit.ly/2J81irm} in \S{}6.

		\textbf{Esempio:}
		\begin{lstlisting}
		spring.mail.host=localhost
		spring.mail.port=1025
		spring.mail.username=
		spring.mail.password=
		spring.mail.properties.mail.smtp.auth=false
		spring.mail.properties.mail.smtp.from=no-reply@example.com
		spring.mail.properties.mail.smtp.starttls.enable=false
		\end{lstlisting}
