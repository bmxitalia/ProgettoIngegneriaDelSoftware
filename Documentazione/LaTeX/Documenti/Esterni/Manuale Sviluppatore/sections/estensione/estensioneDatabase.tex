	\subsection{Implementare un nuovo database}
	\label{newdatabase}

		Per l'implementazione di un nuovo database per la lettura e l'inserimento 
		di trace/metriche/ecc sono richiesti due step: la realizzazione dell'implementazione
		e l'inserimento del nuovo database all'elenco dei driver disponibili.

		\subsubsection{Realizzazione dell'implementazione}

			Per l'implementazione del nuovo database è necessario creare una classe che 
			implementi l'interfaccia \lstinline{it.iks.openapm.databases.Database}.\\
			I metodi richiesti nella classe sono i seguenti:
			\begin{itemize}
				\item \lstinline[columns=fixed, language=Java]{Iterable<Map<String, Object>> findAllInInterval(String index, Date from, Date to)}:\\
					Dato un indice o tabella e un intervallo di tempo rappresentato da una 
					data iniziale \lstinline{from} inclusiva e da una data finale \lstinline{to} esclusiva, 
					ritornare un \lstinline[language=Java]{Iterable} con tutti i valori contenuti in quell'intervallo 
					di tempo.
					
				\item 
				\lstinline[columns=fixed, language=Java]{Iterable<Map<String, Object>> findNewestWhere(String index, Map<String, Object> conditions)}:\\
					Dato un indice o tabella e un insieme di condizioni \lstinline{attributo => valore}, restituire 
					una lista di elementi che corrispondono a quelle condizioni in ordine temporale 
					discendente (i nuovi elementi prima).
					
				\item 
				\lstinline[columns=fixed, language=Java]{void insert(String index, Map<String, Object> object, String type)}:\\
					Dato un indice o tabella, una oggetto e una tipologia, inserire il seguente oggetto 
					nel database.\\
					\textit{Nota:} E' compito di questa funzione aggiungere/sovrascrivere il campo rappresentante 
					la data di inserimento del documento.
					
			
			\end{itemize}


		\subsubsection{Aggiunta del database all'elenco dei driver disponibili}

		Per poter essere utilizzabile come database deve essere disponbile un valore di \lstinline{driver} 
		(vedi §\ref{impostazionidatabase}) per richiamare la classe.
		La selezione viene fatta utilizzando un Bean condizionale di Spring, ed è disponibile un 
		Bean per ogni tipologia di elemento supportata.

		Per aggiungere un bean, creare un metodo in una delle seguenti classi:
		\begin{itemize}
		\item \textbf{Azioni di rimedio:} \lstinline{it.iks.openapm.databases.DatabaseActionsServiceConfig};
		\item \textbf{Baseline:} \lstinline{it.iks.openapm.databases.DatabaseBaselinesServiceConfig};
		\item \textbf{Metriche:} \lstinline{it.iks.openapm.databases.DatabaseMetricsServiceConfig};
		\item \textbf{Trace:} \lstinline{it.iks.openapm.databases.DatabaseTracesServiceConfig}.
		\end{itemize}\\
		
		\textbf{Esempio:}
		Aggiungere il driver \lstinline{mysql} per le azioni di rimedio.

		\begin{lstlisting}[language=Java]
	@Bean
	@ConditionalOnProperty(name = 
		"app.actions.database.driver", havingValue = "mysql")
	public Database actionsDatabase() {
		return new MySQLDatabase(host, port, protocol,
			timestamp, keepalive);
	}
		\end{lstlisting}
