\subsection[Implementare una nuova strategia]{Implementare una nuova strategia per il calcolo delle baseline}
\label{newstrategy}
	\subsubsection{Realizzazione della strategia}
	
		Una strategia è una classe che implementa l'interfaccia \lstinline{it.iks.openapm.strategies.Strategy}.
		È disponibile una classe astratta \lstinline{AbstractStrategy} che implementa il costruttore aspettato e 
		salva la variabile \lstinline[language=Java]{protected final int dataPoints}.
		I metodi necessari alla classe della nuova strategia sono:
		
		\begin{itemize}
	
			\item \lstinline[language=Java]{List<Date> periods(Date calculationTime)}:
				Data una data per il calcolo della baseline, determinare una lista di periodi 
				da cui prelevare le metriche, indicate dalla data di inizio.
				La data passata come parametro \textbf{\underline{\large deve}} essere l'ultima della lista.
				Il numero di periodi è governato dalla variabile \lstinline{dataPoints}.
				Per esempio, in una strategia mensile con \lstinline{dataPoints = 4}, per calcolare i periodi 
				per \lstinline{15:00 - 2 Giugno}, verranno restituite le seguenti date:
				\begin{itemize}
					\item \lstinline{15:00 - 2 Marzo};
					\item \lstinline{15:00 - 2 Aprile};
					\item \lstinline{15:00 - 2 Maggio};
					\item \lstinline{15:00 - 2 Giugno};			
				\end{itemize}
			
			\item \lstinline[language=Java]{Map<String, Object> attributes(Date calculationTime)}
				Data una data per il calcolo della baseline, restituisce un insieme di attributi 
				che verranno aggiunti alla baseline finale. 
				Questi attributi devono essere sufficienti per riconoscere la baseline in base alla strategia.
				Per esempio, in una strategia mensile, per calcolare gli attributi per \lstinline{15:00 - 2 Giugno}, 
				verranno restituiti:
				\begin{itemize}
					\item Il tipo di strategia (mensile);
					\item L'orario di calcolo (15:00);
					\item Il giorno del mese (2);
				\end{itemize}
				
			\item \lstinline[language=Java]{Date previousPeriod(Date calculationTime)}
				Data la data di una metrica, restituire la precedente baseline da utilizzare per il confronto.
				Per esempio, in una strategia mensile, per la metrica del `15:14 - 2 Giugno`, 
				verrà restituita la data \lstinline{15:00 - 2 Maggio}. 

		\end{itemize}
	\subsubsection{Registrare la strategia}

		Per collegare un identificatore alla classe corretta è necessario aggiungere un valore 
		\lstinline[language=Java]{<String, Class<? extends Strategy>>} al campo \lstinline{strategiesTable} 
		di 	\lstinline{it.iks.openapm.strategy.StrategyFactory}. 
		Per farlo è sufficiente aggiungere una riga nell'inizializzazione statica della variabile 
		(poche righe dopo la sua dichiarazione).

		Per esempio, per collegare l'identificatore \lstinline{daily} alla strategia \lstinline{DailyStrategy} è
		sufficiente aggiungere questa linea di codice:
		\begin{lstlisting}[language=Java]
	// StrategyFactory L34
	tempStrategiesTable.put("daily", DailyStrategy.class);
		\end{lstlisting}
		
		
		La modifica della classe \verb=StrategyFactory= alla registrazione di ogni nuova strategia risulta un punto critico in 
		questa classe, la soluzione a questo problema è l'implementazione di meccanismi di Reflection che permetterebbero
		di modificare dinamicamente la classe \verb=StrategyFactory=.\\
		Al momento \GroupName{} non ha ancora implementato questa soluzione.
