\subsection{Implementare un nuovo operatore}
\label{newoperatore}
	\subsubsection{Realizzazione dell'operatore}
		
		Un operatore è una classe che implementa l'interfaccia \lstinline{it.iks.openapm.operators.Operator}
		con i metodi più opportuni.
		Se l'operatore in questione non supporta una determinata operazione, 
		è sufficiente lanciare un'eccezione \lstinline{UnsupportedOperationException}.\\
		I metodi necessari alla classe del nuovo operatore sono:

		\begin{itemize}
			\item \lstinline[language=Java]{boolean valid()}:
				Determina se l'operatore è stato configurato correttamente o meno;

			\item \lstinline[language=Java]{boolean match(Map<String, Object> item)}:
				Dato un oggetto, determina se l'oggetto soddisfa i filtri di questo operatore.
				Per esempio, un operatore di uguaglianza, impostato sui valori \lstinline{attribute = 20}, 
				restituirà vero solo se l'oggetto passato ha il campo \lstinline{attribute} uguale a \lstinline{20};

			\item \lstinline[language=Java]{String group(Map<String, Object> item)}:
				Dato un oggetto, verrà restituita una stringa che rappresenta un ``gruppo'' in cui 
				posizionare l'oggetto.
				Per esempio, un operatore di attributo, impostato sul valore \lstinline{key}, 
				restituirà il valore dell'attributo \lstinline{key} dell'oggetto passato,
				risultato nella raggruppamento degli oggetti a seconda del valore di \lstinline{key};

			\item \lstinline[language=Java]{double calculate(Iterable<Map<String, Object>> items)}:
				Dato un insieme di oggetti, restuire il valore numerico corrispondente.
				Per esempio, un operatore di media, impostato sul valore \lstinline{key}, restituirà 
				la media aritmetica dei valori \lstinline{key} di tutti gli oggetti passati.

		\end{itemize}
		È disponibile una classe astratta \lstinline{it.iks.openapm.operators.AbstractOperator} da estendere,
		che implementa tutti i metodi richiesti lanciando un'eccezione \lstinline{UnsupportedOperationException}. 
		Viene dato di default un comportamento particolare al metodo \lstinline{group}: vengono separati tutti 
		gli oggetti che restituiscono \lstinline{true} all'operazione di \lstinline{match} da quelli che 
		restituiscono \lstinline{false}.
		Lo scopo di questa classe è ridurre il \glossaryItem{boilerplate} code necessario per creare un operatore che non 
		implementa tutte le funzioni.

	\subsubsection{Registrare l'operatore}

		Per collegare un identificatore alla classe corretta è necessario aggiungere un valore 
		\lstinline[language=Java]{<String, Class<? extends Operator>>} al campo \lstinline{operatorsTable} 
		di 	\lstinline{it.iks.openapm.operators.OperatorFactory}. 
		Per farlo è sufficiente aggiungere una riga nell'inizializzazione statica della variabile 
		(poche righe dopo la sua dichiarazione).

		Per esempio, per collegare l'identificatore \lstinline{=} all'operatore \lstinline{EqualsOperator}:
		\begin{lstlisting}[language=Java]
	// OperatorFactory L35
	tempOperatorsTable.put("=", EqualsOperator.class);
		\end{lstlisting}

		La modifica della classe \verb=OperatorFactory= alla registrazione di ogni nuovo operatore risulta un punto critico in 
		questa classe, la soluzione a questo problema è l'implementazione di meccanismi di Reflection che permetterebbero
		di modificare dinamicamente la classe \verb=OperatorFactory=.\\
		Al momento \GroupName{} non ha ancora implementato questa soluzione.
