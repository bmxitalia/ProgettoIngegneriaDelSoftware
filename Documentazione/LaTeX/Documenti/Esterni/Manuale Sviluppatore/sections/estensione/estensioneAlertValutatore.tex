\subsection{Implementare un nuovo valutatore per gli alert}
\label{newvalutatore}
	\subsubsection{Realizzare un nuovo valutatore per gli alert}
	
		Un valutatore è una classe che implementa l'interfaccia \lstinline{it.iks.openapm.alerts.evaluators.}\\
		\lstinline{Evaluator}.
		È disponibile una classe astratta \lstinline{EvaluatorFactory} che implementa il costruttore aspettato.
		I metodi necessari alla classe del nuovo valutatore per gli alert sono:
			
		\begin{itemize}
		
			\item \lstinline[language=Java]{boolean evaluate(List<Condition> conditions)}:
				data una lista di condizioni, ritorna \lstinline{true} se soddisfa il valutatore.
				Ad esempio, avendo un valutatore del tipo ``match su tutte le condizioni'',
				la variabile ritornata sarà \lstinline{true} se esiste un match su tutte le condizioni,
				\lstinline{false} altrimenti. 
		\end{itemize}
	
	\subsubsection{Registrare un nuovo valutatore per gli alert}
	
		Per collegare un identificatore alla classe corretta è necessario aggiungere un valore 
		\lstinline[language=Java]{<String, Class<? extends Evaluator>>} al campo \lstinline{evaluatorsTable} 
		di \lstinline{it.iks.openapm.alerts.factories.}\\ \lstinline{EvaluatorFactory}. 
		Per farlo è sufficiente aggiungere una riga nell'inizializzazione statica della variabile 
		(poche righe dopo la sua dichiarazione).

		Per esempio, per collegare l'identificatore \lstinline{all_match} al valutatore \lstinline{AllMatchEvaluator} è
		sufficiente aggiungere questa linea di codice:
		\begin{lstlisting}[language=Java]
	// EvaluatorFactory L31
	tempEvaluatorsTable.put("all_match", AllMatchEvaluator.class);
		\end{lstlisting}\\
		
		La modifica della classe \verb=EvaluatorFactory= alla registrazione di ogni nuovo valutatore risulta un punto critico in 
		questa classe, la soluzione a questo problema è l'implementazione di meccanismi di Reflection che permetterebbero
		di modificare dinamicamente la classe \verb=EvaluatorFactory=.\\
		Al momento \GroupName{} non ha ancora implementato questa soluzione.