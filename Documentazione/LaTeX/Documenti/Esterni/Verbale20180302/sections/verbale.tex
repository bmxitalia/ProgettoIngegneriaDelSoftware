\section{Informazioni generali}

\subsection{Informazioni incontro}
\begin{itemize}
\item \textbf{Luogo}: \Proponente{}, Corso Stati Uniti 14/bis, Padova (PD);
\item \textbf{Data}: 2 Marzo 2018;
\item \textbf{Ora}: 15:30 - 18:30;
\item \textbf{Componenti interni}: \Tommaso, \Mattia, \Isacco;
\item \textbf{Componenti esterni}: Stefano Bertolin, Stefano Lazzaro.
\end{itemize}

\subsection{Argomenti}
Durante l'incontro sono stati negoziati i requisiti di Machine Learning.

\section{Riassunto incontro}
È stato deciso insieme alla Proponente di rimuovere i requisiti di Machine Learning che erano stati selezionati dal team come opzionali. Il motivo della scelta è dato da un ritardo nella pianificazione dovuto al periodo di sessione di esami invernale. Infatti tutti i componenti sono stati impegnati negli appelli durante questo periodo per cui il team ha preferito dedicarsi ai requisiti fondamentali del progetto. Per cui le ore persona che non verranno impiegate in Machine Learning verranno utilizzare per solidificare i requisiti fondamentali e per garantire l'efficacia dell'intero progetto.

\subsection{Riepilogo decisioni}
\begin{center}
    \begin{tabular}{c | c}
        \centering
        \rowcolor[gray]{.9} { \textbf{Codice} } & { \textbf{Decisione} } \\ 
        \hline
        \rowcolor[gray]{.8} VE\_20180302.1 & Eliminazione dei requisiti opzionali di Machine Learning \\
        \rowcolor[gray]{.9} VE\_20180302.2 & Eliminazione degli UC sul Machine Learning \\
    \end{tabular}
\end{center}
