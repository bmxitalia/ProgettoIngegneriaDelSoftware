\section{B}

    \subsection*{Baseline}

        Nel ciclo di vita di un progetto, essa è punto d'arrivo tecnico dal quale non si retrocede. Un progetto prevede una successione
        di baseline che va gestita con processi dedicati. Identificare bene le baseline durante la pianificazione di un progetto è
        importante per garantire:
        \begin{itemize}
        	\item riproducibilità;
        	\item tracciabilità;
        	\item analisi, valutazione, confronto.
        \end{itemize}
        
        \\
        Una baseline, nel contesto del progetto didattico OpenAPM, è un punto base che viene associato a 
        delle metriche. Se le metriche si scostano troppo dalla baseline viene evidenziata una criticità 
        del sistema, il qualche deve gestire la situazione con un'azione di rimedio.

    \subsection*{Bash}

        Bash è una “Unix shell”, vale a dire un'interfaccia a riga di comando che permette l'interazione con sistemi
        operativi derivati da Unix. Tecnicamente bash è un clone evoluto della shell standard di Unix ed il suo nome
        è l'acronimo di Bourne Again Shell. Si tratta quindi di un interprete di comandi che permette all'utente di
        comunicare col sistema operativo o di eseguire programmi e script.

    \subsection*{Batch}

        Con il termine Batch ci si riferisce ad una serie di procedure automatiche, schedulate a determinati orari.
        
    \subsection*{Benchmark}

		Sono valori di riferimento ottenuti da best practice di dominio. Durante il processo di misurazione si cerca di ottenere risultati tendenti ai benchmark.

    \subsection*{Big Bang integration}

        È un test d'integrazione nel quale tutti i componenti o moduli di sistema vengono integrati e testati contemporaneamente.
        In questo approccio i singoli moduli non vengono integrati fino a quando non hanno superato i corrispondenti test
        d'unità, diventando pronti all'integrazione. Questo test viene eseguito per sapere se tutti i moduli funzionano correttamente
        stando insieme. A causa dell'integrazione che avviene tutta in un unico step, se si verificano degli errori, diventa difficile
        per i programmatori conoscere la causa principale degli errori. In caso di problemi sarà necessario staccare modulo per modulo
        per riuscire a scovare l'errore.
