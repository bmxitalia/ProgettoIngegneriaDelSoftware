\section{T}

	\subsection*{Task}
		Un task è un compito ben definito, di solito assegnato ad una singola persona, che necessita di essere svolto entro un periodo di tempo limitato.

    \subsection*{Team}

        Sinonimi: gruppo di progetto, gruppo \ProjectName. \\
        Un team di progetto è l'insieme delle persone che ricoprono ruoli di un progetto.
        I ruoli sono funzioni aziendali assegnate al progetto. I possibili ruoli per un progetto SW sono:

        \begin{itemize}
            \item Analista;
            \item Progettista;
            \item Programmatore;
            \item Verificatore;
            \item Amministratore;
            \item Responsabile.
        \end{itemize}

    \subsection*{Technology baseline}

        La Technology baseline presenta le tecnologie, i framework e le librerie che si è deciso di utilizzare
        per la realizzazione di un progetto software.
        Oltre alle tecnologie vengono presentate anche le motivazioni alla base delle scelte fatte.

    \subsection*{Template}

        Un template, per quanto riguarda un documento, specifica la sua struttura ed evita la ripetizione di parti comuni.
        Per struttura si intende la formattazione del documento, ovvero del frontespizio, della paginazione,
        di figure e tabelle presenti nel documento.
        Nei documenti presentati il termine viene utilizzato in due contesti:
	\begin{itemize}
		\item template \LaTeX{}: per mantenere una struttura coerente nei documenti;
		\item template per mail: per mantenere una struttura coerente nelle mail mandate come azione di rimedio nel nostro 			progetto.
	\end{itemize}

	\subsection*{Tendenze (serie storiche)}	

		Sono tendenze nel tempo di un fenomeno che vengono calcolate tramite degli indicatori che monitorano l'andamento del fenomeno nel tempo. Vengono utilizzate per valutare se un valore misurato è in linea rispetto alla tendenza temporale.
	
	\subsection*{Trace}

        Una trace, in ambito di monitoring, è un insieme di informazioni relative all'esecuzione di una singola operazione
        o richiesta da parte di un'applicazione.
        Possiamo ritrovare lo stesso concetto, anche se leggermente ridotto di significato, nel termine log.
        
