\section{I}

    \subsection*{Indice Gulpease}

        L'Indice Gulpease è un indice di leggibilità di un testo tarato sulla lingua italiana definito nel 1988
        nell'ambito delle ricerche del GULP (Gruppo Universitario Linguistico Pedagogico) presso l'Università degli
        studi di Roma "La Sapienza". Rispetto ad altri ha il vantaggio di utilizzare la lunghezza delle parole in
        lettere anziché in sillabe, semplificando il calcolo automatico di questo indice di leggibilità.

    \subsection*{Indice di Flesch}

        L'indice di Flesch, o formula di Flesch è un indice di leggibilità tarato sulla lingua inglese che nel 1972 è
        stato rivisto ed adattato alla lingua italiana. L'adattamento di Roberto Vacca e Valerio Franchina sarà utile
        alla creazione dell'Indice Gulpease. Questo indice, a differenza dell'indice Gulpease, sfrutta il numero totale
        di sillabe su un campione di 100 parole.

    \subsection*{Ipertestuale}

        Per navigazione ipertestuale di intende la possibilità di navigare in un documento tramite link presenti
        nell'indice. Questi link puntano alle sezioni del documento.
        Nei documenti senza navigazione ipertestuale si richiede l'utilizzo di una navigazione sequenziale.
        Una navigazione sequenziale non permette di raggiungere una sezione tramite un link sull'indice ma rende
        necessario scorrere tutto il documento fino alla sezione voluta.