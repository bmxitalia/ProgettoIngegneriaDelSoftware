\section{A}

    \subsection*{Agent}

        In informatica un Agent è un programma che agisce per un utente esterno o per un altro programma con una relazione
        di ``agency". In parole povere, un Agent si può chiamare anche Bot.
        Uno degli esempi più semplici di Agent è l'assistente personale di Apple, Siri.

    \subsection*{Agile}

        Le metodologie Agile sono un insieme di metodi e tecniche atte a sviluppare software.
        Queste metodologie derivano principalmente dal ``Manifesto per lo sviluppo agile del software", scritto
        nel 2001 da alcuni famosi informatici.
        Esse si focalizzano sulla consegna di software in tempi brevi e frequentemente.
        Fra le tecniche consigliate dai metodi agili troviamo:

        \begin{itemize}
            \item team di sviluppo piccoli, cross-funzionali e auto-organizzativi;
            \item sviluppo iterativo e incrementale;
            \item coinvolgimento diretto e continuo del cliente nel processo di sviluppo.
        \end{itemize}

    \subsection*{APM}

        Con APM, acronimo di Application Performance Management, ci si riferisce al monitoraggio delle performance di
        una applicazione software.
        Viene utilizzato principalmente per diagnosticare i problemi di performance che non mantengono il ``level of
        service" atteso.
        
    \subsection*{Attori}

        Nell'Ingegneria del software, il termine attore viene utilizzato all'interno dei casi d'uso per indicare il
        soggetto che interagisce con il sistema.
