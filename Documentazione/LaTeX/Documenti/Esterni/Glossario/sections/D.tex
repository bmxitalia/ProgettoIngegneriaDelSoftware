\section{D}

    \subsection*{Design Pattern}

        Un Design Pattern rappresenta una soluzione generale ad un problema ricorrente.
        Alcuni esempi sono:

        \begin{itemize}
            \item \textbf{Abstract factory} - fornisce un'interfaccia per creare famiglie di oggetti connessi o
                                                dipendenti fra loro;
            \item \textbf{Adapter} - converte l'interfaccia di una classe in una interfaccia diversa;
            \item \textbf{MVC - Model View Controller} - pattern architetturale che consiste nel separare i componenti
                                                            della logica di businness dalla logica di presentazione e
                                                            da quelli che tali funzionalità utilizzano.
        \end{itemize}

    \subsection*{DevOps}

        In informatica, DevOps rappresenta una metodologia di sviluppo software.
        DevOps prevede che il team di sviluppo software lavori anche alle altre operazioni non inerenti al solo
        sviluppo, come ad esempio logistica e ricerca.

    \subsection*{Docker}

        Docker è uno strumento open-source che automatizza l'utilizzo di applicazioni all'interno di container software. Docker utilizza le funzionalità di isolamento delle risorse
        del kernel Linux per consentire a container indipendenti di coesistere sulla stessa istanza di Linux, evitando l'installazione e la manutenzione di una macchina virtuale.

        I namespace del kernel Linux per lo più isolano ciò che l'applicazione può vedere dell'ambiente operativo, mentre i cgroups forniscono l'isolamento delle risorse, inclusa
        la CPU, la memoria, i dispositivi di I/O.
