\section{P}

    \subsection*{Periodo di slack}

        Per periodo di slack si intende il tempo durante il quale un'attività può essere ritardata,
        senza ritardare l'intero progetto di cui fa parte.

    \subsection*{Physical SLOC}

        Source lines of code o più brevemente LOC, è una metrica software che misura le dimensioni di un software
        basandosi sul numero di linee di codice sorgente. Physical SLOC non è altro che un metodo di calcolo.
        In questa misurazione si contano tutte le righe di testo del codice sorgente includendo anche i commenti e le
        linee bianche, se la loro percentuale non supera il 25\% delle linee.

	\subsection*{Policy}

	    Una policy, in informatica, indica una serie di regole che devono essere rispettate, per fare in modo che i
	    soggetti controllati da essa presentino le stesse caratteristiche.

    \subsection*{Prodotto}

        Per prodotto si intende un bene materiale o immateriale, risultato di un'attività di progetto.

    \subsection*{Product baseline}

        La Product baseline rappresenta la baseline architetturale del prodotto, deve mostrare coerenza con quanto
        dichiarato nella Technology Baseline.
        Ne fanno parte diagrammi delle classi e di sequenza, comprensivi della contestualizzazione dei design
        pattern adottati, all'interno dell'architettura del prodotto.

    \subsection*{Progetto}

        Per progetto nell'ambito dell'Ingegneria del Software si intende il progetto SW.
        Un progetto è un insieme di attività e compiti con le seguenti proprietà:

        \begin{itemize}
            \item devono raggiungere determinati obiettivi con specifiche fissate;
            \item hanno date di inizio e fine fissate;
            \item possono contare su limitate disponibilità di risorse, ad esempio, persone, tempo e fondi;
            \item consumano risorse nel loro svolgersi.
        \end{itemize}

    \subsection*{Proof-of-Concept}

        La Proof-of-Concept è la dimostrazione del funzionamento di base di un applicativo o di un intero sistema.
        Un esempio è il prototipo, che presenta solo le funzionalità di base.

    \subsection*{Proponente}

        Persona o azienda che propone un capitolato d'appalto riguardante un progetto SW.
