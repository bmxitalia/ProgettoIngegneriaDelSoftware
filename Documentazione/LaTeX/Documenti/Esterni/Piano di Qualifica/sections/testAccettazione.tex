\subsection{Specifica dei test} \label{sec:test}
	\subsubsection{Test di accettazione}\label{testAccettazione}
		A seguito di un incontro con la \glossaryItem{Proponente} \Proponente{} è stata richiesta loro l'identificazione dei Test di Accettazione.

		Essi consistono nelle procedure che un utente deve poter seguire sul prodotto software finito affinché
		lo si possa validare rispetto alle esigenze contrattuali descritte nel Capitolato.
	
		Questi test verranno svolti sotto la supervisione della Proponente e dei \glossaryItem{Committenti} durante il Collaudo, la
		cui data prevista è il 15 giugno 2018, in concomitanza con la Revisione di Accettazione.

		\myparagraph{Test TAFO1} L’amministratore di sistema deve poter avviare una procedura Batch secondo configurazioni temporali prestabilite. \\
		L'utente può, modificando il file '\textit{application.properties}', configurare i vari parametri di esecuzione; le istruzioni sul come effettuare la configurazione sono disponibili nel \textit{Manuale Utente v2.0.0}. A quel punto si potrà proseguire generando e lanciando il file \textit{.jar} che avvierà la procedura batch.

		\myparagraph{Test TAFO2} La procedura batch deve essere in grado di generare una metrica. \\
		Una volta avviata la procedura batch, questa inizierà l'attività di monitoraggio delle traces nel database, raggruppandole e calcolando su di esse le metriche per poi salvarle in un indice, proprio come specifa la configurazione fornita. I risultati potranno dunque essere osservati ispezionando l'indice del database. 

		\myparagraph{Test TAFO3} L’inserimento di una metrica deve scatenare la creazione di una baseline basata su tale metrica, da parte della procedura batch. \\
		La procedura batch procederà in automatico a creare o aggiornare le baseline corrispettive delle metriche calcolate. Anche in questo caso i risultati sono osservabili ispezionando l'indice configurato per il salvataggio.

		\myparagraph{Test TAFD5} Allo scattare di un critical event, la procedura batch, deve poter inviare una e-mail di notifica. \\
		Il test verrà eseguito in questo modo:		
		\begin{itemize}
			\item stress dell'entità monitorata;
			\item verifica del lancio di un critical event;
			\item verifica della ricezione della e-mail.
		\end{itemize}

		\myparagraph{Test TAFD6} Allo scattare di un critical event, la procedura batch, deve poter memorizzare il critical event. \\
		Il test verrà eseguito in questo modo:		
		\begin{itemize}
			\item stress dell'entità monitorata;
			\item verifica del lancio di un critical event;
			\item verifica del salvataggio delle informazioni del critical event su indice del database.
		\end{itemize}

		\myparagraph{Test TAFD7} Allo scattare di un critical event, la procedura batch, deve poter eseguire una procedura automatica. \\
		Il test verrà eseguito in questo modo:		
		\begin{itemize}
			\item stress dell'entità monitorata;
			\item verifica del lancio di un critical event;
			\item verifica dell'esecuzione della procedura automatica.
		\end{itemize}
