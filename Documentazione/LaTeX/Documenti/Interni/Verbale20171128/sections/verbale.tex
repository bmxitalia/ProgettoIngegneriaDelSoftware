\section{Informazioni generali}
\subsection{Informazioni incontro}
\begin{itemize}
\item \textbf{Luogo}: Aula 2BC30 Torre Archimede;
\item \textbf{Data}: 28 Novembre 2017;
\item \textbf{Ora}: 14:30 - 18:00;
\item \textbf{Componenti interni}: \Tommaso, \Carlo, \Mattia, \Luca, \Cristian, \Isacco;
\item \textbf{Componenti esterni}: assenti.
\end{itemize}

\subsection{Argomenti}
Durante l'incontro sono state prese decisioni in merito alle tecnologie da utilizzare per il coordinamento del gruppo, 
per la stesura della documentazione, per il tracciamento dei requisiti, e per il controllo di versione. Si è deciso come dividersi i compiti per il Piano di Progetto e il Piano di Qualifica. Si è inoltre creata la mail del gruppo e si è scelto 
il capitolato d'appalto per cui svolgere il progetto.

\section{Riassunto incontro}
Il gruppo ha scelto il capitolato \textbf{C7} \ProjectName{} proposto da \Proponente{} in seguito a diversi sondaggi tenuti 
sull'applicazione \glossaryItem{Slack}. Il gruppo aveva tenuto un incontro con l'azienda SIAV in merito al 
capitolato \textbf{C4}, ma dopo un'attenta analisi dei rischi è stato selezionato il capitolato \textbf{C7} come prima scelta.
Non è stato redatto un verbale per tale incontro in quanto informale e il capitolato è stato scartato.
L'analisi dei rischi si può trovare nel documento \StudioFattibilita.\\

Per il coordinamento del gruppo si sono scelte le seguenti applicazioni:
\begin{itemize}
\item \textbf{Slack}: per un utilizzo puramente informativo e decisionale;
\item \glossaryItem{\textbf{Asana}}: per l'assegnamento dei \glossaryItem{ticket} durante il progetto.
\end{itemize} \\
Per la stesura della documentazione è stato scelto il linguaggio \glossaryItem{\LaTeX{}}.\\
Si è deciso di dividere inizialmente il lavoro nel seguente modo, in base alle disponibilità temporali dei membri:
\begin{itemize}
	\item Piano di Progetto: \Tommaso, \Leonardo, \Cristian;
    \item Piano di Qualifica: \Leonardo, \Isacco, \Carlo, \Mattia, \Cristian, \Tommaso.
\end{itemize}
Per il controllo di versione è stato scelto il servizio \glossaryItem{GitHub}.\\
In seguito a queste scelte si è creata la \glossaryItem{repository} del progetto su GitHub ed il workspace Slack e Asana per il gruppo.\\
Si è inoltre deciso di utilizzare l'applicazione web \glossaryItem{SWEgo} per il tracciamento dei requisiti.
Infine si è creata la mail del gruppo \GroupEmail.
\subsection{Riepilogo decisioni}
\begin{center}
\begin{tabular}{c | c}
\centering
\rowcolor[gray]{.9} {\textbf{Codice}}&{\textbf{Decisione}}\\ \hline
\rowcolor[gray]{.8} VI\_20171128.1 & Scelta del capitolato d'appalto \textbf{C7} \ProjectName \\
\rowcolor[gray]{.9} VI\_20171128.2 & Scelta dell'applicazione Slack \\
\rowcolor[gray]{.8} VI\_20171128.3 & Scelta dell'applicazione Asana \\
\rowcolor[gray]{.9} VI\_20171128.4 & Scelta del linguaggio \LaTeX{} \\
\rowcolor[gray]{.8} VI\_20171128.5 & Piano di Qualifica - \Cristian \\
\rowcolor[gray]{.9} VI\_20171128.6 & Piano di Qualifica - \Carlo \\
\rowcolor[gray]{.8} VI\_20171128.7 & Piano di Qualifica - \Leonardo \\
\rowcolor[gray]{.9} VI\_20171128.8 & Piano di Qualifica - \Isacco \\
\rowcolor[gray]{.8} VI\_20171128.9 & Piano di Qualifica - \Mattia \\
\rowcolor[gray]{.9} VI\_20171128.10 & Piano di Qualifica - \Tommaso \\
\rowcolor[gray]{.8} VI\_20171128.12 & Piano di Progetto - \Leonardo \\
\rowcolor[gray]{.9} VI\_20171128.13 & Piano di Progetto - \Tommaso \\
\rowcolor[gray]{.8} VI\_20171128.14 & Piano di Progetto - \Cristian \\
\rowcolor[gray]{.9} VI\_20171128.15 & Scelta del servizio GitHub \\
\rowcolor[gray]{.8} VI\_20171128.16 & Scelta dell'applicazione web SWEgo \\
\rowcolor[gray]{.9} VI\_20171128.17 & Creazione mail milctdev.team@gmail.com\\
\end{tabular}
\end{center}