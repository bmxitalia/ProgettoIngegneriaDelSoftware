\section{Informazioni generali}
\subsection{Informazioni incontro}
\begin{itemize}
\item \textbf{Luogo}: Torre Archimede;
\item \textbf{Data}: 19 Marzo 2018;
\item \textbf{Ora}: 11:30 - 13:00;
\item \textbf{Componenti interni}: \Tommaso, \Carlo, \Isacco, \Mattia, \Luca, \Cristian, \Leonardo;
\item \textbf{Componenti esterni}: assenti.
\end{itemize}

\subsection{Argomenti}
Durante l'incontro è stata analizzata la situazione di avanzamento al termine della revisione di progetto e le cause del calo della resa del gruppo negli ultimi mesi.

\section{Riassunto incontro}
È stata individuata una certa difficoltà di più membri del gruppo nel rispettare le scadenze delle task ad essi assegnate. Ogni membro ha condiviso le proprie difficoltà ed è stato rinnovato l'impegno di rispettare le scadenze nel proseguimento del progetto. Ci si impegna a fare ciò cercando di suddividere le task in micro-compiti e comunicando più frequentemente con i membri del gruppo per verificare lo stato di avanzamento delle varie task e per collaborare in caso di difficoltà. Inoltre è stato deciso di discutere maggiormente, anche attraverso il canale di comunicazione Slack, le scelte effettuate durante le varie attività, per permettere a tutti i membri del gruppo di partecipare a tutti gli aspetti dello sviluppo del progetto.

Sono stati assegnati i seguenti task:
\begin{itemize}
	\item correzione generale dei documenti:
	\begin{itemize}
		\item inserire riferimenti più precisi in tutti i documenti: \Isacco{}.
		\item rendere leggibile il registro delle modifiche: \Carlo{};
	\end{itemize}
	\item incremento Norme di Progetto V2:
	\begin{itemize}
		\item espandere le norme relative alla progettazione: \Carlo{}.
	\end{itemize}
	\item correzione Piano di Qualifica V2:
	\begin{itemize}
		\item spostare le sezioni segnalate in revisione di avanzamento nei documenti corretti: \Leonardo{}.
	\end{itemize}
	\item correzione e incremento Glossario V2
	\begin{itemize}
		\item trovare e correggere le voci spurie presenti nel glossario: \Isacco{};
	\end{itemize}
	\item correzione e incremento Piano di Progetto V2:
	\begin{itemize}
		\item attualizzazione dei rischi per progettazione in dettaglio: \Mattia{};
		\item correzione punteggiatura documento: \Mattia{};
		\item inserire consuntivo nuovo periodo: \Mattia{};
		\item aggiornare preventivo a finire: \Mattia{};
	\end{itemize}
	\item correzione Analisi Requisiti V2:
	\begin{itemize}
		\item analisi del documento: \Cristian{};
		\item rendere le funzioni del sistema più chiare ed esaustive possibili: \Cristian{};
		\item scrivere in modo più chiaro gli UC segnalati alla revisione di avanzamento: \Tommaso{}.
	\end{itemize}
\end{itemize}

\subsection{Riepilogo decisioni}

\begin{center}
    \begin{tabular}{c | p{11cm}}
        \centering
        \rowcolor[gray]{.9} { \textbf{Codice} } & { \textbf{Decisione} } \\ 
        \hline
        \rowcolor[gray]{.8} VI\_20180319.1 & correzione generale dei documenti - \Isacco , \Carlo \\
        \rowcolor[gray]{.9} VI\_20180319.2 & incremento Norme di Progetto - \Carlo \\
        \rowcolor[gray]{.8} VI\_20180319.3 & correzione Piano di Qualifica - \Leonardo \\
        \rowcolor[gray]{.9} VI\_20180319.4 & correzione e incremento Glossario - \Isacco \\
        \rowcolor[gray]{.8} VI\_20180319.5 & correzione e incremento Piano di Progetto - \Mattia \\
        \rowcolor[gray]{.9} VI\_20180319.6 & correzione Analisi Requisiti - \Cristian , \Tommaso \\
    \end{tabular}
\end{center}
