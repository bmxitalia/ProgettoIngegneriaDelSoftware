\subsection{Capitolato C3}

\subsubsection{Descrizione generale}
Il capitolato C3 propone la realizzazione di un'interfaccia grafica per \glossaryItem{Speect} che agevoli l'ispezione del suo stato interno durante il funzionamento.

\subsubsection{Obbiettivo finale}
L'obbiettivo finale è realizzare un interfaccia grafica per Speect che funzioni come un debugger specializzato. 
Durante la generazione dell'audio, l'interfaccia deve mostrare all'utente il comportamento di tutti i plugin in funzione all'interno di Speect.
Inoltre deve essere disponibile la possibilità di esportare lo stato attuale di Speect per poterlo utilizzare in futuro nell'automatizzazione di test.

\subsubsection{Tecnologie richieste}
\begin{itemize}
\item \textbf{Speect}, sistema multilingua di \glossaryItem{text-to-speech};
\item \textbf{Qt}, librerie per lo sviluppo di interfacce in C++.
\end{itemize}

\subsubsection{Valutazione finale}
La natura puramente tecnica del capitolato ha portato il team a scartarlo. 
Fattore fondamentale è stata la necessità di lavorare utilizzando il linguaggio C++ ed il toolkit Qt, 
che non suscitano alcun interesse da parte dei membri del team poiché già sviscerati durante il corso di studi.
