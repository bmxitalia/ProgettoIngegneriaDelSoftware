\subsection{Capitolato C4}

\subsubsection{Descrizione generale}
Il capitolato C4 propone la realizzazione di un sistema di suggerimenti integrato ai servizi gestionali offerti dalla Proponente, che permetta agli utilizzatori del software di ricevere elementi correlati durante l'utilizzo del programma.

\subsubsection{Obbiettivo finale}
L'obbiettivo finale è realizzare un software in grado di integrarsi con un sistema gestionale fornito dalla Proponente.
Un utente che utilizza questo sistema, durante le azioni quotidiane, deve ricevere insieme alle email, ai profili e agli articoli, un gruppo di documenti correlati allo scopo di trovare rapidamente tutte le informazioni necessarie.

\subsubsection{Tecnologie richieste}
\begin{itemize}
\item \textbf{Apache Solr}, sistema di ricerca di documenti;
\item \textbf{Keycloak}, sistema di gestione di identità e permessi;
\item \textbf{Evernote}, servizio online di gestione e condivisione di appunti e file.
\end{itemize}

\subsubsection{Valutazione finale}
Il team ha mostrato grande interesse nella proposta del capitolato; la possibilità di lavorare ad un motore di ricerca è stato un elemento di grande peso nelle decisioni. 
Altra nota apprezzata è stata la possibilità di integrare sistemi di Machine Learning.
In conclusione, questo capitolato non è stato scelto perché la necessità di lavorare principalmente su cellulare è stata considerata di scarso interesse e di eccessiva complessità. 
A contribuire alla scelta è stata anche l'autoformazione di Machine Learning richiesta, giudicata troppo impegnativa in relazione ai requisiti richiesti, nonostante fosse di grande interesse per vari membri del gruppo.
