\section{Informazioni generali}
\subsection{Informazioni incontro}
\begin{itemize}
\item \textbf{Luogo}: Torre Archimede;
\item \textbf{Data}: 7 Aprile 2018;
\item \textbf{Ora}: 10:00 - 12:30;
\item \textbf{Componenti interni}: \Tommaso, \Mattia, \Isacco;
\item \textbf{Componenti esterni}: assenti.
\end{itemize}

\subsection{Argomenti}
Durante l'incontro si è discusso della pianificazione del team per il periodo di progettazione in dettaglio. Il gruppo ha assegnato i task per la progettazione e la codifica del prodotto finale.

\section{Riassunto incontro}
Durante l'incontro sono stati assegnati i task per la stesura della product baseline e per la codifica del prodotto finale. Sono stati assegnati i seguenti compiti:
\begin{enumerate}
	\item progettazione in dettaglio: costruzione dei diagrammi UML di classe e di sequenza utili a illustrare un'architettura matura del prodotto. Per la creazione dei diagrammi si è scelto il programma Visual Paradigm 15.0 Community Edition: \Tommaso{}, \Mattia{}, \Luca{}, \Leonardo{}, \Cristian{};
	\item Product Baseline: stesura di tale allegato tecnico. Coloro che si occuperanno di tale compito dovranno essere necessariamente 
	affiancati dai Progettisti e dai Programmatori al fine di preparare una presentazione efficace: \Isacco{}, \Cristian{}, \Carlo{};
	\item codifica del prodotto: codifica del prodotto finale con i relativi test di unità, di integrazione e di sistema: \Mattia{}, \Luca{}, \Carlo{}, \Isacco{}, \Cristian{}.
\end{enumerate}
In questo periodo si è deciso di concentrarsi sul prodotto e di occuparsi dei documenti da sistemare e incrementare non appena la codifica sarà terminata.

\end{itemize}

\subsection{Riepilogo decisioni}

\begin{center}
    \begin{tabular}{c | p{12cm}}
        \centering
        \rowcolor[gray]{.9} { \textbf{Codice} } & { \textbf{Decisione} } \\ 
        \hline
        \rowcolor[gray]{.8} VI\_20180412.1 & Progettazione dell'architettura in dettaglio - \Tommaso{}, \Mattia{}, \Luca{}, \Leonardo{}, \Cristian{} \\
        \rowcolor[gray]{.9} VI\_20180407.2 & Creazione dei diagrammi di classe - \Tommaso \\
        \rowcolor[gray]{.8} VI\_20180407.3 & Scelta di Visual Paradigm 15.0 Community Edition per la realizzazione dei diagrammi \\
        \rowcolor[gray]{.9} VI\_20180407.4 & Creazione dei diagrammi di sequenza - \Mattia \\
        \rowcolor[gray]{.8} VI\_20180407.5 & Verifica dei diagrammi UML - \Leonardo{}, \Luca{}, \Tommaso{} \\
        \rowcolor[gray]{.9} VI\_20180407.6 & Stesura parte diagrammi di classe nella Product Baseline - \Isacco{} \\
        \rowcolor[gray]{.8} VI\_20180407.7 & Stesura parte diagrammi di sequenza nella Product Baseline - \Cristian{} \\
        \rowcolor[gray]{.9} VI\_20180407.8 & Stesura parte sui design pattern nella Product Baseline - \Carlo{} \\
        \rowcolor[gray]{.8} VI\_20180407.9 & Codifica del secondo incremento del prodotto finale - \Mattia{} \\
        \rowcolor[gray]{.9} VI\_20180407.10 & Codifica dei test per il secondo incremento - \Luca{} \\
    \end{tabular}
\end{center}
