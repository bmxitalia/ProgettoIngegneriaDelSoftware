\section{Capitolato scelto}

\subsection{Descrizione generale}
Il capitolato C7 descrive \textit{OpenAPM}, strumento di \glossaryItem{APM} basato su tecnologie open source che 
fornisca le informazioni necessarie per diagnosticare la propria applicazione utilizzabile dalla parte di monitoring della pratica DevOps.

\subsection{Obbiettivo finale}
Il \glossaryItem{progetto} propone due obbiettivi a scelta.
Il primo obbiettivo è sviluppare un sistema di visualizzazione basato su Kibana con lo scopo di mostrare i dati raccolti da un Agent. 
Il secondo è sviluppare un sistema di analisi \glossaryItem{batch} degli stessi dati, in grado di estrarre statistiche ed informazioni utili a DevOps.

\subsection{Tecnologie richieste}
\begin{itemize}
\item \textbf{ElasticSearch}, motore di ricerca per grandi quantità di dati;
\item \textbf{Kibana}, sistema di visualizzazione dei dati contenuti all'interno di Elasticsearch;
\item \textbf{D3.js}, libreria Javascript per realizzare grafici.
\end{itemize}

\subsection{Valutazione finale}
Il team considera stimolante l'utilizzo di tecnologie rinomate come Elasticsearch e Kibana e ritiene 
che queste avranno un peso importante nel bagaglio formativo dei membri. 
Molto rilevante nella scelta è stata la disponibilità e la strumentazione fornita dai \glossaryItem{Proponenti}.
Inoltre è stata apprezzata la scelta di basarsi su tecnologie Open Source.
