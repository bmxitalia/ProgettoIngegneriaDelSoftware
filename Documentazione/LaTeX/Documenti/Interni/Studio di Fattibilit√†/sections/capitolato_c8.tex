\subsection{Capitolato C8}

\subsubsection{Descrizione generale}
Il capitolato C8 propone la realizzazione di una piattaforma Web che consenta agli artisti emergenti di gestire i propri tour
tramite la visualizzazione di tutti i locali nelle vicinanze che siano compatibili con l'ospitarne una tappa e fungere anche da intermediario nella trattativa.

\subsubsection{Obbiettivo finale}
L'obbiettivo finale è realizzare una piattaforma che permetta agli artisti emergenti di mettersi in contatto con i gestori di luoghi o locali in cui è possibile esibirsi.
Ad un artista deve essere possibile inserire un percorso o selezionare una zona e visualizzare tutti i luoghi disponibili.
In dettaglio devono essere implementate le seguenti funzioni:
\begin{itemize}
\item pagina di registrazione di un artista, con tutti i collegamenti e le informazioni desiderabili (sito personale, profilo social, email, ecc);
\item pagina di registrazione di un locale/spazio;
\item profilo di ogni artista e locale;
\item funzionalità di ricerca per zona;
\item funzionalità di ricerca per destinazione/percorso;
\item sistema per permettere ad un locale di invitare un artista;
\item chat per la comunicazione tra artista e locale;
\item permettere ad un utente pubblico di consultare gli eventi per area geografica;
\item sistema di feedback per artisti e locali;
\item gestione dei pagamenti.
\end{itemize}

\subsubsection{Tecnologie richieste}
\begin{itemize}
\item \textbf{React}, una libreria Javascript per realizzare interfacce utente.
\end{itemize}

\subsubsection{Valutazione finale}
Sebbene la semplicità del capitolato sia stata considerata come fattore positivo, il gruppo ha deciso di scartarlo. 
Si chiede di sviluppare troppe funzioni rispetto al tempo a disposizione.
Inoltre le tecnologie richieste non sono state considerate stimolanti da parte del team.
