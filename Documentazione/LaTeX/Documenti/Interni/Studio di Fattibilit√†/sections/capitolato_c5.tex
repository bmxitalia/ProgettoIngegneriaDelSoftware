\subsection{Capitolato C5}

\subsubsection{Descrizione generale}
Il capitolato C5 propone la realizzazione di Ironworks, un sistema di generazione di codice a partire da \glossaryItem{Robustness Diagrams}.

\subsubsection{Obbiettivo finale}
L'obbiettivo finale è realizzare un editor per scrivere Robustness Diagrams. 
Il programma deve essere in grado di generare il codice Java a partire dai Robustness Diagrams inseriti.
Deve essere generato anche il codice SQL per la creazione di database o lo script di configurazione di un \glossaryItem{ORM}.

\subsubsection{Tecnologie richieste}
\begin{itemize}
\item \textbf{ArgoUML}, applicazione per la realizzazione di Robustness Diagrams;
\item \textbf{Hibernate}, ORM in ambiente Java.
\end{itemize}

\subsubsection{Valutazione finale}
L'obbiettivo del capitolato non è risultato interessante per i membri del team.
In più i tempi di previsti di autoformazione sulle tecnologie sono stati valutati come eccessivi.
