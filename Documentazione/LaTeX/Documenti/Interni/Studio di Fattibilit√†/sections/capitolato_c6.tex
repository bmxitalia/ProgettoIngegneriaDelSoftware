\subsection{Capitolato C6}

\subsubsection{Descrizione generale}
Il capitolato C6 propone la realizzazione di un sistema di gestione universitario basato su blockchain.

\subsubsection{Obbiettivo finale}
L'obbiettivo finale è realizzare un sistema di gestione universitario in grado di gestire: \textit{Università}, \textit{Docenti} e \textit{Studenti}.
I componenti dovranno essere in grado di compiere le seguenti azioni:
\begin{itemize}
\item l'\textit{Università} dovrà poter creare i corsi disponibili ogni anno. Ogni corso conterrà un insieme di esami disponibili. Ogni esame avrà un argomento, un punteggio in crediti ed un \textit{Docente} associato;
\item il \textit{Docente} dovrà essere in grado di specificare il voto di ogni \textit{Studente} iscritto ad un suo esame;
\item lo \textit{Studente} dovrà poter visualizzare il suo rendimento ed iscriversi agli esami disponibili.
\end{itemize}

\subsubsection{Tecnologie richieste}
\begin{itemize}
\item \textbf{Ethereum}, piattaforma per lo sviluppo di applicazioni basate su blockchain;
\item \textbf{React}, una libreria Javascript per realizzare interfacce utente.
\end{itemize}

\subsubsection{Valutazione finale}
Il team non ha mostrato nessun interesse nella tecnologia blockchain e nella gestione di un sistema universitario. 
Un'indagine più approfondita delle tecnologie richieste ha ulteriormente scartato il capitolato.
