\section{Altri capitolati}
\subsection{Capitolato C1}

\subsubsection{Descrizione generale}
Il capitolato C1 propone lo sviluppo di un'applicazione di \glossaryItem{Machine Learning} in grado di ascoltare gli incontri giornalieri sullo stato di avanzamenti dei loro progetti, comprendere i dialoghi ed estrarne gli argomenti emersi.

\subsubsection{Obbiettivo finale}
L'obbiettivo del progetto è realizzare un'applicazione che estrapoli gli argomenti emersi durante gli incontri giornalieri e li renda disponibili per una consultazione veloce in futuro.
Per raggiungere questo obbiettivo si propone il seguente workflow:
\begin{itemize}
\item registrare le conversazioni;
\item trasformare l'audio in testo utilizzando i servizi di \glossaryItem{Google Cloud Platform};
\item utilizzare un algoritmo di Machine Learning per estrapolare gli argomenti dalla trascrizione delle conversazioni;
\item visualizzare gli argomenti in un interfaccia Web.
\end{itemize}
Il Proponente richiede inoltre un'analisi di efficienza dei vari servizi di \glossaryItem{Speech-to-Text} messi a disposizione da \glossaryItem{GCP}, \glossaryItem{AWS} o altri proposti dal gruppo.

\subsubsection{Tecnologie richieste}
\begin{itemize}
\item \textbf{Google Natural Language API}, strumento per ottenere informazioni chiave da un testo;
\item \textbf{Node.js}, runtime per programmi Javascript;
\item \textbf{Twitter Bootstrap}, libreria per la realizzazione di interfacce Web.
\end{itemize}

\subsubsection{Valutazione finale}
Il team considera la complessità richiesta, sia da un punto di vista progettuale sia materiale, eccessiva per il limitato monte ore disponibile. 
Inoltre non è valutato di alcun interesse realizzare un sistema di trascrizione da utilizzare internamente alla Proponente.
