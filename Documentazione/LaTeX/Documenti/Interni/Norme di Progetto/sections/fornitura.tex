\section{Processi primari}

	\subsection{Fornitura} \label{fornitura}

		\subsubsection{Scopo del processo}

			Il processo di fornitura contiene tutte le attività e i compiti che i membri del gruppo
			\GroupName{} sono tenuti a rispettare al fine di proporsi e diventare Fornitori nei confronti
			della \glossaryItem{Proponente} \Proponente{} e dei \glossaryItem{Committenti} \Committenteinriga{}.

			A contratto stipulato vengono determinate le procedure e le risorse necessarie alla gestione dello sviluppo
			del prodotto \ProjectName{}, tra cui la stesura del \vPianoDiProgetto{} e la consegna del prodotto realizzato.
	
		\subsubsection{Studio di Fattibilità}

			Il documento \vStudioDiFattibilita{} deve contenere la valutazione tecnica sui capitolati proposti
			e le motivazioni che hanno portato alla scelta del progetto da realizzare.

			Il \Responsabile{} ha il compito di organizzare riunioni tra i componenti del gruppo per discutere dei capitolati
			e discutere le opinioni di ogni componente.
			Successivamente l'\Analista{} deve redigere il documento indicando per ogni capitolato:

			\begin{itemize}
				\item \textbf{Descrizione generale}: breve presentazione del progetto;
				\item \textbf{Obiettivo finale}: descrizione in linea di massima delle caratteristiche principali richieste nel prodotto
					completato ed il suo ambito di utilizzo;
				\item \textbf{Tecnologie richieste}: elenco delle tecnologie da impiegare nella realizzazione del prodotto;
				\item \textbf{Valutazione finale}: riassunto degli aspetti principali che hanno portato il gruppo ad accettare o scartare
					il capitolato in questione.
			\end{itemize}

		\subsubsection{Rapporti con la Proponente}

			Durante l'intero svolgimento del progetto il gruppo intende instaurare con la Proponente \Proponente{},
			in particolar modo nelle figure dei referenti Stefano Bertolin e Stefano Lazzaro, un rapporto di collaborazione
			costante e costruttiva al fine di:

			\begin{itemize}
				\item determinarne i bisogni;
				\item individuare norme consone all'esecuzione dei processi;
				\item ricevere feedback riguardo all'andamento del progetto;
				\item stimare costi e tempi;
				\item accordarsi circa la qualifica di prodotto ultimato.
			\end{itemize}
			
			Le modalità di comunicazione sono descritte in §\ref{comunicazione_esterna}.
		
		\subsubsection{Documentazione fornita}

			Al fine di assicurare la massima trasparenza circa le attività progettuali, verranno distribuiti alla Proponente \Proponente{}
			ed ai Committenti \Committenteinriga{} i documenti descritti in §\ref{documenti_prodotti}.
			
			In particolare, in vista della Revisione di Accettazione del 15 giugno 2018 verranno consegnati, con annessa Lettera di
			Presentazione, i seguenti documenti:

			\begin{itemize}
				\item \vNormeDiProgetto{};
				\item \vAnalisiDeiRequisiti{};
				\item \vPianoDiProgetto{};
				\item \vPianoDiQualifica{};
				\item \textit{Product Baseline v1.0.0};
				\item \textit{Manuale Utente v2.0.0};
				\item \textit{Manuale Sviluppatore v2.0.0};
				\item i \textit{Verbali} stilati nello trascorrere dalla scorsa revisione.
			\end{itemize}

		\subsubsection{Consegna e collaudo del prodotto}

			Al fine di validare il prodotto ultimato rispetto agli obblighi contrattuali, il gruppo deve effettuare un
			collaudo sotto la supervisione della Proponente e dei Committenti.

			La data prevista è il 15 giugno 2018 in concomitanza con la Revisione di Accettazione e per la quale è stato richiesto al Fornitore
			di stilare i test di accettazione da eseguirvisi.

			Inoltre, il gruppo si impegnerà a consegnare entro il giorno lavorativo precedente il prodotto finito, su supporto DVD,
			comprensivo di: utilità di installazione, istruzioni per l'uso, sorgenti completi, utilità di compilazione, documentazione
			ed eventuali utilità di collaudo.
			
			In preparazione di ciò il gruppo deve assicurare la correttezza, 	completezza ed affidabilità di ogni parte del
			prodotto realizzato affinché si possa dimostrare che:

			\begin{itemize}
				\item tutti i requisiti obbligatori descritti in \vAnalisiDeiRequisiti{} siano completamente soddisfatti;
				\item l'esecuzione di tutti i test descritti nel \vPianoDiQualifica{} abbia esito positivo.
			\end{itemize}
			
		\subsubsection{Completamento del progetto}
		
			A seguito della consegna del prodotto e del superamento del collaudo, il progetto si intende concluso.

			Ciò significa che, salvo diversi accordi con la Proponente \Proponente{}, il gruppo \GroupName{} non seguirà l'attività
			di manutenzione del prodotto \textit{OpenAPM}.
