\subsection{Verifica} \label{verifica}

	\subsubsection{Scopo del processo}

		Il processo di verifica è un processo che determina se i prodotti software di
		un'attività soddisfano i requisiti o le condizioni imposte ad essi nelle attività 
		precedentemente svolte.	
		Per costi ed efficacia delle prestazioni, la verifica dovrebbe essere integrata, 
		quanto prima possibile, nel processo che la impiega.
			
	\subsubsection{Analisi}	

		\myparagraph{Analisi statica}

            \'E una tecnica che studia il codice e la documentazione e ne verifica la conformità alle regole,
            l'assenza di difetti e la presenza di proprietà positive.
            Questa tecnica non richiede esecuzione del prodotto software in alcuna sua parte per cui è essenziale
            finché il sistema non è completamente disponibile.
		    \'E attuabile tramite due tecniche:

                \begin{itemize}

                    \item \textbf{Walkthrough}:
                        attività che richiede la collaborazione di più persone per effettuare una lettura a largo
                        spettro di tutto il documento e il codice in esame, con lo scopo di rivelare la presenza
                        di difetti.
                        Ogni difetto riscontrato verrà discusso tra Verificatore e autore; è importante che ci
                        sia una terza figura che mantenga il controllo della discussione, detta arbitro.
                        Per ogni documento deve essere redatta una lista di controllo con i difetti rilevati e le
                        decisioni prese.
                        Le fasi del walkthrough sono le seguenti:

                            \begin{itemize}
                                \item pianificazione;
                                \item lettura;
                                \item discussione;
                                \item correzione dei difetti.
                            \end{itemize}

                        Ognuna delle precedenti fasi elencate deve essere documentata;

                    \item \textbf{Inspection}:
                        attività normalmente svolta da una sola persona che effettua una lettura mirata e
                        strutturata del documento, volta a localizzare gli errori segnalati nella lista
                        di controllo; tramite controlli ripetuti verrà progressivamente
                        ampliata per rendere più efficace l'attività di inspection.
                        Le fasi di inspection sono:

                            \begin{itemize}
                                \item pianificazione;
                                \item definizione della lista di controllo;
                                \item lettura;
                                \item correzione dei difetti.
                            \end{itemize}

                        Ognuna delle precedenti fasi elencate deve essere documentata.

                \end{itemize}

            Essendo walkthrough una tecnica poco efficiente verrà impiegata principalmente nella prima parte del progetto.
            Nelle fasi successive si utilizzerà la lista di controllo prodotta da walkthrough e dalle segnalazioni
            dei committenti per effettuare inspection.

        \myparagraph{Analisi dinamica}

            \'E una tecnica di analisi del prodotto software che richiede l'esecuzione del programma.
            Vengono	effettuati dei test su parti del sistema per verificare che ognuna di esse produca
            il risultato desiderato. \\
            Prima di eseguire un qualsiasi test è necessario conoscere la precondizione e postcondizione
            del codice analizzato per poterne decretare l'esito finale.
            Ci sono diversi tipi di test:

                \begin{itemize}
                    \item \textbf{Test di Unità}:
                        va ad isolare la parte più piccola di software testabile nell'applicazione,
                        chiamata unità, per stabilire se essa funziona esattamente come previsto.
                        Per effettuare test di unità è necessaria la scrittura di codice fittizio.
                        Può essere di due tipologie:

                            \begin{itemize}
                                \item \textbf{Driver}: è un software che guida il test d'unità sostituendo
                                l'unità chiamante per testare l'unità chiamata;
                                \item \textbf{Stub}: è un software che simula il comportamento delle unità
                                chiamate per testare un'unità chiamante.
                            \end{itemize}

                        Ogni unità deve essere sottoposta a test prima di poter essere integrata con
                        quelle già testate. Inoltre, ogni test di unità dovrà avere un codice identificativo
                        che segue la seguente nomenclatura:
                        \[TU = [\mbox{Numero progressivo}]\]
                        
                    \item \textbf{Test di Integrazione}:
                        prevede la combinazione di unità già testate in un unico componente per verificare
                        che il loro funzionamento integrato abbia esito atteso.
                        Richiede una strategia di integrazione incrementale che può essere di due tipologie:

                            \begin{itemize}
                                \item \textbf{Bottom-up}: si sviluppano e si integrano prima le unità con minore
                                dipendenza funzionale e maggiore utilità e poi si risale l'albero delle
                                dipendenze;
                                \item \textbf{Top-down}: si sviluppano prima le unità più esterne poste sulle
                                foglie dell'albero delle dipendenze e poi si scende l'albero.
                            \end{itemize}\\
						Ogni test di integrazione prevede un codice che lo identifica così formato:
						\[TI = [\mbox{Numero progressivo}]\]
                    \item \textbf{Test di Regressione}:
                        questo test deve essere eseguito ad ogni modifica ad un implementazione del sistema.
                        Vengono ripetuti sul codice modificato i test precedentemente utilizzati, per
                        verificare che le modifiche effettuate non abbiano alterato elementi precedentemente
                        funzionanti.
                        I contenuti di tale test devono essere decisi nel momento in cui si approvano modifiche
                        al software.
                        Ogni test di regressione prevede un codice che lo identifica così formato:
						\[TR = [\mbox{Numero progressivo}]\]
                    \item \textbf{Test di Sistema}:
                        ha inizio con il completamento del test d'integrazione e verifica il comportamento
                        dinamico del sistema completo rispetto ai requisiti software.
                        Ogni test di sistema dovrà avere un requisito correlato da testare e dovrà rispettare 
                        la seguente nomenclatura:
                        \[TS = [\mbox{Tipologia}][\mbox{Importanza}][\mbox{Codice}]\]
                                               
                        Dove Tipologia e Importanza sono dedotti dal requisito correlato che il test andrà a verificare, 
                        mentre Codice è un codice composto da una serie di numeri separati tramite
                   		punto che identificano il test in maniera univoca e lo esprimono gerarchicamente;
                    \item \textbf{Test di Accettazione}:
                        prevede il collaudo del prodotto in presenza della \glossaryItem{Proponente}.
                        Il superamento del test permette il rilascio ufficiale del prodotto sviluppato.
                        
                \end{itemize}
		
	\subsubsection{Strumentazione}

		\begin{itemize}
			\item \textbf{Hunspell}:
				Hunspell è un correttore ortografico integrabile con TexMaker e che supporta dizionari italiani;
			\item \textbf{Julia}:
				Julia è un analizzatore statico per Java basato sulla tecnica scientifica dell'interpretazione astratta che garantisce la precisione e l'affidabilità dei suoi risultati.
		\end{itemize}