\section{Guida all'uso di Git} \label{guidaGit}

                Questa appendice contiene una serie di comandi Git che permetterà ai membri del team di
                progetto di interagire con \glossaryItem{GitHub}:

                    \begin{itemize}
                        \item \textbf{\textit{git clone}}: comando che permette di clonare il repository
                        remoto su GitHub in locale.
                        La sintassi è \textit{git clone indirizzoRepository} dove indirizzoRepository è
                        l'indirizzo raggiungibile tramite il menù a tendina ``Clone or Download" su GitHub;
                        \item \textbf{\textit{git add nomeFile}}: comando che aggiunge il file alla lista
                        dei file tracciati da Git, contenuti nell'index.
                        Il comando deve essere utilizzato per aggiungere nuovi file o per proporre la modifica
                        di file già esistenti in una versione successiva. Se si modifica un file e non si
                        esegue questo comando, le modifiche non verranno apportate nella versione successiva.
                        Il comando può essere usato per aggiungere una cartella oppure si può eseguire
                        \textit{git add .} per aggiungere tutti i file, comprese le sotto cartelle;
                        \item \textbf{\textit{git pull}}: comando che permette di aggiornare il contenuto
                        del repository locale con il contenuto di quello remoto. In questo modo eventuali
                        modifiche attuate da altre persone verranno riportate sul repository locale e si
                        può lavorare sui file più recenti;
                        \item \textbf{\textit{git status}}: comando che visualizza lo stato dei file,
                        dividendoli in tre gruppi:

                            \begin{enumerate}
                                \item \textbf{File non tracciati}: mostra i file che non sono mai stati inseriti
                                nel controllo versione di Git;
                                \item \textbf{File modificati ma non aggiornati}: mostra i file modificati,
                                rispetto al commit precedente, che non sono stati aggiunti tramite
                                \textit{git add} e che quindi non verranno inseriti nel prossimo commit;
                                \item \textbf{File modificati pronti per il commit}: mostra i file modificati,
                                rispetto al commit precedente, che sono stati aggiunti tramite \textit{git add}
                                e che quindi sono pronti per il commit.
                            \end{enumerate}

                        \item \textbf{\textit{git diff}}: mentre \textit{git status} mostra i file che sono stati
                        modificati, per entrare nei dettagli delle righe modificate è necessario
                        utilizzare \textit{git diff}.
                        Il comando può essere utilizzato in due modalità:

                            \begin{enumerate}
                                \item \textbf{\textit{git diff}}: mostra le righe che sono state cambiate nei
                                file non ancora preparati per la commit, confrontati con la copia
                                presente nell'ultima commit;
                                \item \textbf{\textit{git diff --staged}}: mostra le righe che sono state
                                cambiate nei file preparati per la commit, confrontati con la
                                copia presente nell'ultima commit.
                            \end{enumerate}

                        \item \textbf{\textit{git commit -m ``Messaggio commit"}}: comando che permette di
                        rendere definitivi i file aggiunti o modificati tramite il comando \textit{git add}.
                        Se non si lancia il comando \textit{git commit} e si esegue \textit{git push}, non
                        verranno caricate eventuali modifiche o file aggiunti al repository locale sul
                        repository remoto. \textit{-m} è l'opzione per inserire un messaggio significativo
                        (obbligatorio) per far comprendere ai membri del team la tipologia di modifiche che
                        sono state effettuate;
                        \item \textbf{\textit{git log}}: comando che permette di visualizzare lo storico
                        dei commit;
                        \item \textbf{\textit{git checkout}}: comando che permette di fare cose differenti
                        a seconda di come viene utilizzato:

                            \begin{enumerate}
                                \item \textbf{\textit{git checkout -b nomeBranch}}: crea un nuovo
                                branch chiamato ``nomeBranch". Una volta creato il branch, tutti
                                i comandi che verranno lanciati saranno eseguiti su quel branch;
                                \item \textbf{\textit{git checkout nomeBranch}}: permette di spostarsi sul
                                branch ``nomeBranch".
                            \end{enumerate}

                        \item \textbf{\textit{git merge nomeBranch}}: comando che permette di riportare le modifiche apportate a ``nomeBranch" nel branch attualmente selezionato.
                        Il comando deve essere lanciato dal branch master;
                        \item \textbf{\textit{git push}}: comando che aggiorna il repository remoto
                        sovrascrivendolo con il repository locale;
                        \item \textbf{configurazioni utente}: se si tratta della prima volta che si utilizza
                        Git allora al primo tentativo di commit verrà lanciato un errore relativo al fatto
                        che non si sono settati il nome utente e la e-mail.
                        Per far questo si utilizzano i comandi:

                            \begin{enumerate}
                                \item \textbf{\textit{git config --global user.name ``nomeUtente"}}: comando che
                                permette di configurare Git con il nostro nome. In questo modo ogni qualvolta che
                                si effettuerà una commit, il nostro nome verrà associato alla commit;
                                \item \textbf{\textit{git config --global user.email ``e-mail"}}: comando identico
                                al precedente ma che permette di configurare l'indirizzo e-mail.
                            \end{enumerate}

                    \end{itemize}