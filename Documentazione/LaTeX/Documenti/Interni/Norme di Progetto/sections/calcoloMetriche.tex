\section{Calcolo ed uso delle metriche}
\label{metricheDettaglio}

	In questa sezione si provvederà alla descrizione delle metriche, presenti nella §\ref{subsec:metriche}, 
	che permettono di quantificare e valutare la qualità dei processi e dei prodotti di \GroupName{}.
	All'interno della spiegazione di ogni metrica verrà illustrato quando, come e su cosa viene fatta 
	la misurazione durante il processo di verifica.

	\subsection{Misure e metriche per i processi}

		\subsubsection{Schedule Variance}
			La Schedule Variance è un indice di efficienza che ha come oggetto la durata temporale 
			di un processo o di un'attività. Questa metrica aiuta il Responsabile di Progetto nella 
			creazione dei prospetti orari inseriti nei consuntivi di periodo, e di conseguenza aiuta 
			il team nell'analisi dell'utilizzo di risorse temporali.

			Il calcolo della Schedule Variance avviene in questo modo:
			\[SV = \mbox{\textit{data conclusione reale}} - \mbox{\textit{data conslusione preventivata}}\]


			Entrambe le date si riferiscono alla conclusione dell'attività o del processo.

		\subsubsection{Cost Variance}
			La Cost Variance, o Variazione di Costo, è una metrica che analizza il costo, nonché le risorse 
			legate ad un processo o ad un attività. Essa può essere influenzata anche dalla metrica sopracitata.

			La Variazione di Costo viene così calcolata:
			\[ CV = \mbox{\textit{costo delle risorse effettivo}} - \mbox{\textit{costo delle risorse preventivato}}\]


		\subsubsection{SPICE}
			Al termine di ogni periodo, il \glossaryItem{team} \GroupName{} provvederà alla valutazione della 
			qualità dei processi tramite lo standard ISO/IEC 15504 conosciuto come SPICE.
			Lo standard SPICE, ed i livelli di maturità, vengono illustrati in maniera completa ed approfondita 
			nell'Appendice \ref{app:standard}.

\newpage

	\subsection{Misure e metriche per i prodotti}
		\subsubsection{Misure e metriche per i documenti}
		%----------------
		\myparagraph{Indice Gulpease}
			Per analizzare la leggibilità della documentazione prodotta, il team \GroupName{} ha deciso di 
			avvalersi dell'\glossaryItem{indice Gulpease}. Questo è stato creato per venire incontro alla 
			complessità della lingua italiana, non contemplata in altri indici, come l'\glossaryItem{indice di Flesch}.

			L'indice Gulpease viene calcolato tramite questa formula:
			\[ IG = 89 + \frac{(300 \times \mbox{\textit{numero delle frasi}}) - (10 \times \mbox{\textit{numero delle lettere}})}{\mbox{\textit{numero delle parole}}} \]

			Il valore ottenuto indicherà la leggibilità del testo e può variare da $0$, indice di bassissima 
			leggibilità, a $100$, indice di ottima leggibilità.

	\subsubsection{Misure e metriche per il software}
		Al fine di poter correttamente quantificare e valutare la qualità del prodotto software, il 
		team \GroupName{} ha deciso utilizzare diverse metriche. Gran parte delle metriche indicate in 
		questa sezione verranno riviste ed aggiornate nel corso dei successivi periodi.

		\myparagraph{Grado di accoppiamento}
			Questa metrica indica una misura la dipendenza di un componente del prodotto con le altri parti di \ProjectName{}.\\
			Per questo calcolo si utilizza:
			\begin{description}
				\item[Structural Fan-Out (SFOUT): ] 
					il grado di accoppiamento efferente permette di avere una 
					visione del numero di moduli che usufruiscono della componente oggetto di analisi.
					Il valore di questo indice è semplicemente dato dal conteggio delle componenti indicate 
					poco sopra. Un valore molto basso può indicare una scarsa utilità del modulo analizzato, 
					all'opposto un grado troppo alto potrebbe indicare un pericoloso livello di dipendenza.
				
				\item[Structural Fan-In (SFIN): ]
					il grado di accoppiamento afferente, così come l'accoppiamento efferente, ha come oggetto di 
					analisi il numero di moduli che sono legati alla componente in analisi.
					Questa volta si prende in considerazione il numero di moduli esterni che vengono utilizzati.
					Un indice ottimale dovrebbe avere un valore di $0$ o $1$, questo perché minore è il suo valore, 
					più il modulo è indipendente ai cambiamenti del resto del sistema. Un valore eccessivamente alto 
					è indice di troppa dipendenza rispetto al resto del sistema.
			\end{description}

		\myparagraph{Code Coverage}
			Il code coverage è una metrica che, sfruttando una misura detta Logical Source Lines of Code (Logical SLOC), 
			indica la percentuale di statements coperti dai test.\\
			
			La Logical SLOC è una metrica che dà un idea della grandezza del prodotto software contando il numero di linee di codice. 
			Il team ha scelto di utilizzare la variante definita Logical SLOC, andando quindi a contare solamente 
			il numero di \glossaryItem{statements} all'interno del codice.\\			

			Il valore della code coverage è così calcolato:
			\[ CC = \frac{\mbox{\textit{Numero di statement coperti da test}}}{\mbox{\textit{Logical SLOC}}} \times 100 \]
			% CC = SLOC / Num. Statement coperti da test * 100

		\myparagraph{Rapporto linee di commento per linee di codice}
			Un indice di buona manutenibilità del codice potrebbe essere il rapporto tra \glossaryItem{Physical SLOC} 
			e numero di linee di commento all'interno dello stesso.

			Il valore viene espresso in percentuale e viene così calcolato:
			\[ RLCLC = \frac{\mbox{\textit{Numero di linee di commento}}}{\mbox{\textit{Numero di linee di codice totali}}} \times 100 \]
			% RLCLC = Num. linee di codice totali / Num. linee di commento * 100

		\myparagraph{Complessità ciclomatica}
			L'indice di complessità di un programma aiuta ad identificare il numero di test necessari al raggiungimento 
			di un coverage completo. Questa metrica software può essere applicata anche a packages, moduli, metodi o classi.

			Il calcolo avviene sfruttando il \glossaryItem{grafo di controllo di flusso} e l'indice non è altro che 
			il numero di cammini indipendenti attraverso il codice sorgente. La formula è quindi la seguente:
			\[ v(G) = e - n + p \]
			% v(G)=e-n+2p

			Dove:
			\begin{itemize}
				\item \textbf{n:} è il numero di nodi del grafo, nonché il numero di tutti i gruppi indivisibili di istruzioni;
				\item \textbf{e:} rappresenta il numero di archi del grafo, cioè il numero di collegamenti tra due 
				nodi tali che, il nodo seguente possa essere eseguito immediatamente dopo il nodo preso di riferimento;
				\item \textbf{p:} è il numero di componenti connesse.
			\end{itemize}

		\myparagraph{Percentuale superamento test}
			Questa metrica indica quanti dei test implementati hanno esito positivo e può essere ottenuta così:
			\[ PST = \frac{\mbox{\textit{Numero test superati}}}{\mbox{\textit{Numero test implementati}}} \times 100 \]


		\myparagraph{Requisiti obbligatori soddisfatti}
			Questa metrica aiuta il team a capire in che quantità sono stati soddisfatti i requisiti obbligatori 
			indicati in'\vAnalisiDeiRequisiti{}.

			Il valore viene espresso in percentuale e viene calcolato come segue:
			\[ROS = \frac{\mbox{\textit{Num. Requisiti obbligatori soddisfatti}}}{\mbox{\textit{Num. Requisiti obbligatori individuati}}}\times 100\]
