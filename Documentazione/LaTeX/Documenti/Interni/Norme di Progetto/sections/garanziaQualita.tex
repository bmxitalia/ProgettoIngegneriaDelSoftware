\subsection{Garanzia di qualità} \label{garanzia di qualita}

    In questa sezione vengono definite le norme e la struttura delle metriche e degli obiettivi di qualità, descritti
    nel \vPianoDiQualifica{}. Metriche ed obiettivi devono essere decise dagli Amministratori in collaborazione con i Verificatori.

    \subsubsection{Notazione per la classificazione}

        Sia gli obiettivi di qualità che le metriche devono essere classificati secondo la seguente notazione:

            \begin{center}
                \textbf{$\{$Classe$\}\{$Tipo$\}\{$Oggetto$\}^*\{$codice\_identificativo$\} : \{$Nome$\}$}
            \end{center}

            \begin{itemize}
                \item \textbf{Classe:} indica se si tratta di un obiettivo di qualità o una metrica. Può essere:

                    \begin{itemize}
                        \item \textbf{O:} per un obiettivo;
                        \item \textbf{M:} per una metrica.
                    \end{itemize}

                \item \textbf{Tipo:} stabilisce se riguarda un prodotto o un processo e può assumere i valori:

                    \begin{itemize}
                        \item \textbf{PD:} per gli obiettivi di prodotto;
                        \item \textbf{PC:} per gli obiettivi di processo.
                    \end{itemize}

                \item \textbf{Oggetto:} nel caso di obiettivi o metriche di prodotto, indica se si riferisce alla parte
                software oppure ad un documento. Può essere:

                    \begin{itemize}
                        \item \textbf{D:} per i documenti;
                        \item \textbf{S:} per il software.
                    \end{itemize}

                \item \textbf{codice\_identificativo:} codice numerico incrementale necessario per l'identificazione;

                \item \textbf{Nome:} titolo che descrive l'obiettivo o la metrica.
            \end{itemize}
	
	\subsubsection{Obiettivi di qualità} \label{obiettivi_qualita}
		In questa sezione vengono illustrati gli obiettivi che \GroupName{} intende 
		raggiungere per assicurare la qualità di processo e di prodotto per quanto 
		riguarda la realizzazione di \ProjectName{}. Inoltre, per  ognuno di questi obiettivi, 
		vengono fissate metriche per rendere quantificabile il raggiungimento della qualità di 
		processo e di prodotto; queste sono descritte nella §\ref{subsec:metriche}.

		\myparagraph{Qualità di processo}
		
			Per realizzare un prodotto valido, \GroupName{} ha deciso di adottare lo standard 
			ISO/IEC 15504 per valutare la qualità di ogni processo necessario allo sviluppo di 
			\ProjectName{}. Viene inoltre utilizzato il ciclo di Deming per assicurare un 
			miglioramento continuo dei processi, senza eventuali regressioni. Nell'appendice \ref{app:standard}
			vengono approfonditi questo metodo e lo standard utilizzato.\\			
			
			Gli obiettivi fissati per i processi sono:
				\begin{itemize}
					\item rispettare tempi e costi descritti nel \vPianoDiProgetto{};
					\item avere prestazioni sempre misurabili;
					\item perseguire un miglioramento continuo delle stesse.
				\end{itemize}

		\myparagraph{Qualità di prodotto}
		
			Basandosi sullo standard ISO/IEC 9126, descritto nell'appendice \ref{app:standard}, 
			sono stati fissati obiettivi che mirano a garantire la qualità del prodotto finale. Questi sono:
				\begin{itemize}
					\item i \textbf{documenti} devono:
					\begin{itemize}
						\item essere leggibili e comprensibili a chiunque;
						\item essere corretti dal punto di vista ortografico, sintattico, semantico e logico.
					\end{itemize}
					\item il \textbf{software} deve:
					\begin{itemize}
						\item soddisfare tutti i requisiti obbligatori descritti in \vAnalisiDeiRequisiti{};
						\item garantire usabilità e manutenibilità;
						\item essere affidabile.	
					\end{itemize}
				\end{itemize}

		\myparagraph{Tabella degli obiettivi}
		
			Viene qui riassunto ogni obiettivo, classificandolo con il suo codice identificativo e 
			indicando le metriche che ne quantificano il raggiungimento. Per una 
			descrizione delle metriche vedere nella sezione §\ref{subsec:metriche}. \\

			\begin{center}
			\begin{longtable}{ | >{\centering\arraybackslash}m{2cm} 
							   | >{\centering\arraybackslash}m{5.5cm} 
							   | >{\centering\arraybackslash}m{5.9cm} | }
        
        	\hline
        		\textbf{ID} & \textbf{Nome} & \textbf{Metrica} \\ \hline
        	\endhead
			        
				OPC1 & Coerenza con \PianoProgetto{} & 	MPC1:Schedule Variance\par
			        									MPC2:Cost Variance
			        									\\ \hline
				OPC2 &  Miglioramento continuo & MPC3:SPICE\\ \hline
				OPDD1 & Leggibilità documenti & MPDD1:Indice Gulpease\\ \hline
				OPDS1 & Implementazione requisiti obbligatori & MPDS6:Requisiti obbligatori soddisfatti\\ \hline
				OPDS2 & Manutenibilità e usabilità & 	MPDS1:Grado di accoppiamento\par
			        									MPDS2:Code coverage\par
			        									MPDS3:Rapporto linee di commento per linee di codice\par
			        									MPDS4:Complessità ciclomatica
			        									\\ \hline
				OPDS3 & Affidabilità & MPDS5:Percentuale superamento test\\ \hline
			\end{longtable}
			\captionof{table}{Tabella degli obiettivi}
			\end{center}\\

			Ogni obiettivo si riterrà raggiungo solamente al raggiungimento del valore minimo
			di ogni metrica che concorre alla quantificazione del suo grado di raggiungimento. 

	\subsubsection{Metriche e misure}
		\label{subsec:metriche}

		Ogni processo ed ogni prodotto dovrebbero sempre presentare un set di \glossaryItem{KPI} 
		che permettano il tracciamento, la comunicazione ed il miglioramento della loro qualità.\\
		In questa sezione pertanto, si provvederà alla presentazione delle metriche che permettano
		di quantificare e valutare la qualità dei processi e dei prodotti di \GroupName{}.
		Le spiegazioni e le modalità di calcolo di ogni metrica sono definite nell'appendice \ref{metricheDettaglio}.\\
		
		Per valutare gli esiti ottenuti dall'utilizzo delle metriche, sono stati individuati tre
		diversi range di risultati possibili che indicano ciascuno un diverso grado di raggiungimento 
		dell'obiettivo di qualità. Questi sono:

		\begin{itemize}
			\item \textbf{Valore negativo:} il valore della misurazione in questo caso viene anche 
			definito inaccettabile, in quanto non soddisfa la qualità minima desiderata;
			\item \textbf{Valore minimo:} valori al di sopra di questa soglia possono essere accettati, 
			ma saranno oggetto di ulteriore analisi in vista di un miglioramento desiderato;
			\item \textbf{Valore ottimale:} rappresenta il valore indice di qualità, da mantenere nel tempo.
		\end{itemize}\\
		
		Le soglie di accettazione per ogni metrica sono descritti nel \vPianoDiQualifica{}.\\		
		

		\myparagraph{Tabella delle metriche}
		
			Nella seguente tabella vengono indicati, oltre a Identificativo e Nome, anche 
			l'obiettivo a cui si riferisce.\\

			\begin{center}
			\begin{longtable}{ | >{\centering\arraybackslash}m{2.5cm} 
							   | >{\raggedright\arraybackslash}m{4cm} 
							   | >{\raggedright\arraybackslash}m{5.7cm} 
							   |  }
        
        	\hline
        		\textbf{ID} & \textbf{Nome} & \textbf{Obiettivo} \\ \hline
        	\endhead
			        
				MPC1 & Schedule Variance & OPC1:Coerenza con Piano di Progetto \\ \hline
			        											
			    MPC2 & Cost Variance & OPC1:Coerenza con Piano di Progetto \\ \hline
			      															
				MPC3 & SPICE & OPC2:Miglioramento continuo \\ \hline

				MPDD1 & Indice Gulpease & OPDD1:Leggibilità documenti \\ \hline

				MPDS1 & Grado di accoppiamento  & OPDS2:Manutenibilità e usabilità \\ \hline

			    MPDS2 & Code coverage  & OPDS2:Manutenibilità e usabilità \\ \hline

			    MPDS3 & Rapporto linee di commento per linee di codice  & OPDS2:Manutenibilità e usabilità \\ \hline

			    MPDS4 & Complessità ciclomatica  &  OPDS2:Manutenibilità e usabilità \\ \hline

				MPDS5 & Percentuale superamento test & OPDS3:Affidabilità \\ \hline

				MPDS6 & Requisiti obbligatori soddisfatti & 
						OPDS1:Im\-ple\-men\-ta\-zio\-ne requisiti obbligatori\\ \hline
															
			\end{longtable}
			\captionof{table}{Tabella delle metriche}
			\end{center}\\            
