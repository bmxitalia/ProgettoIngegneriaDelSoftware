\section{Liste di controllo}
\label{listeControllo}

	Vengono qui presentate le liste di controllo da utilizzare nell'effettuare inspection.

\subsection{Documenti}

	\begin{longtable}{| m{4cm} | c |}
	\hline
		\Gape[0.4cm][0.4cm]{\textbf{Documento}}
		&
		\textbf{Lista di controllo}
	\hline
		Tutti i documenti
		&
		\begin{minipage}[c]{0.7\textwidth}
			\vspace{0.2cm}
			\begin{itemize}
				\item conformità delle date al formato standard;
				\item conformità degli elenchi puntati;
				\item conformità nell'uso delle maiuscole nei titoli;
				\item correttezza ortografica;
				\item correttezza della rappresentazione degli accenti in \LaTeX{};
				\item correttezza della spaziatura prima della punteggiatura;
				\item correttezza delle versioni dei documenti riferiti;
				\item presenza della data di ultima consultazione delle fonti web.
			\end{itemize}
			\vspace{0.2cm}
		\end{minipage}
	\hline
		\AnalisiRequisiti{}
		&
		\begin{minipage}[c]{0.7\textwidth}
			\vspace{0.2cm}
			\begin{itemize}
    				\item conformità alla struttura dei casi d'uso;
    				\item correttezza diagrammi casi d'uso;
    				\item correttezza enumerazione casi d'uso;
    				\item verificabilità dei requisiti.
			\end{itemize}
			\vspace{0.2cm}
		\end{minipage}
	\hline
		\PianoQualifica{}
		&
		\begin{minipage}[c]{0.7\textwidth}
			\vspace{0.2cm}
			\begin{itemize}
				\item assenza di contenuti sovrapposti alle \NormeProgetto{}.
			\end{itemize}
			\vspace{0.2cm}
		\end{minipage}
	\hline
		\NormeProgetto{}
		&
		\begin{minipage}[c]{0.7\textwidth}
			\vspace{0.2cm}
			\begin{itemize}
				\item assenza di descrizioni di procedure in stile narrativo.
			\end{itemize}
			\vspace{0.2cm}
		\end{minipage}
	\hline
		\Glossario{}
		&
		\begin{minipage}[c]{0.7\textwidth}
			\vspace{0.2cm}
			\begin{itemize}
				\item assenza di voci spurie.
			\end{itemize}
			\vspace{0.2cm}
		\end{minipage}
    \hline
	\caption[Lista di controllo per i documenti]{Lista di controllo per i documenti}
	\end{longtable}

\newpage

\subsection{Diagrammi UML} \label{listeUML}

	\begin{longtable}{| m{4cm} | c |}
	\hline
		\Gape[0.4cm][0.4cm]{\textbf{Diagramma}}
		&
		\textbf{Lista di controllo}
	\hline
		Diagrammi delle classi
		&
		\begin{minipage}[c]{0.7\textwidth}
			\vspace{0.2cm}
			\begin{itemize}
				\item presenza della marcatura \textless \textless interface\textgreater \textgreater {}  se la classe è un'interfaccia;
				\item presenza della marcatura \textless \textless enumeration\textgreater \textgreater {} se la classe è un'enumerazione;
				\item presenza della marcatura \textless \textless create\textgreater \textgreater {} in caso di Factory pattern;
				\item separazione tra nome della classe, attributi e metodi;
				\item ogni attributo deve avere un nome e un tipo, scritti nella corretta sintassi;
				\item ogni metodo deve avere un tipo di ritorno, ad eccezione del costruttore;
				\item se un metodo ha una lista di parametri, allora ogni parametro deve avere nome e tipo, scritti nella corretta sintassi;
				\item utilizzo corretto dei modificatori d'accesso (+, -, \#);
				\item descrizione di design pattern tramite commenti con colorazione rossa;
				\item colorazione arancione per classi e package esterni (framework o API);
				\item l'impiego di classi esterne al package implica la modellazione tramite diagrammi di package;
				\item sottolineatura per metodi e attributi statici;
				\item dicitura in corsivo per le classi astratte;
				\item dicitura maiuscola delle costanti di classe;
				\item utilizzo della corretta sintassi per le classi parametriche;
				\item esplicitazione del tipo con cui viene istanziata una classe parametrica nella dipendenza;
				\item utilizzo del corretto tipo di relazione tra le classi;
				\item verificare che non siano stati introdotti concetti legati al software.
			\end{itemize}
			\vspace{0.2cm}
		\end{minipage}
	\hline
		Diagrammi di attività
		&
		\begin{minipage}[c]{0.7\textwidth}
			\vspace{0.2cm}
			\begin{itemize}
				\item confluire di flussi mediante nodo di \textit{merge};
				\item presenza freccia entrante nei nodi di \textit{timeout}.
			\end{itemize}
			\vspace{0.2cm}
		\end{minipage}
	\hline
		Diagrammi di sequenza
		&
		\begin{minipage}[c]{0.7\textwidth}
			\vspace{0.2cm}
			\begin{itemize}
				\item presenza di un nome per ogni istanza di classe;
				\item presenza di una barra di attivazione se un oggetto è attivo;
				\item utilizzo della freccia piena se un messaggio è una chiamata a un metodo di un oggetto;
				\item utilizzo del corretto operatore nei frame di iterazione;
				\item presenza della marcatura \textless \textless create\textgreater \textgreater {} in caso di messaggio di creazione di un oggetto;
				\item presenza della marcatura \textless \textless destroy\textgreater \textgreater {} in caso di messaggio di distruzione di un oggetto;
				\item utilizzo della corretta sintassi per i vari tipi di messaggi.
			\end{itemize}
			\vspace{0.2cm}
		\end{minipage}
	\hline
	\caption[Lista di controllo per i diagrammi UML]{Lista di controllo per i diagrammi UML}
	\end{longtable}

\subsection{Codice} \label{listeCodice}

	\begin{longtable}{| m{4cm} | c |}
	\hline
		\Gape[0.4cm][0.4cm]{\textbf{Parte del software}}
		&
		\textbf{Lista di controllo}
	\hline
		Tutto il codice
		&
		\begin{minipage}[c]{0.7\textwidth}
			\vspace{0.2cm}
			\begin{itemize}
				\item utilizzo dello stile di default di IntellJ IDEA;
				\item ogni metodo pubblico deve avere un JavaDoc con una breve descrizione;
				\item ogni classe deve avere un JavaDoc con una breve descrizione;
				\item i metodi auto-generati devono essere in coda alle classi;
				\item i metodi auto-generati devono essere opportunamente segnalati;
				\item non devono essere specificate versioni ridondanti delle dipendenze esterne.
			\end{itemize}
			\vspace{0.2cm}
		\end{minipage}
	\hline
		Spring
		&
		\begin{minipage}[c]{0.7\textwidth}
			\vspace{0.2cm}
			\begin{itemize}
				\item la configurazione XML di Spring non deve essere usata.
			\end{itemize}
			\vspace{0.2cm}
		\end{minipage}
	\hline
		Spring Batch
		&
		\begin{minipage}[c]{0.7\textwidth}
			\vspace{0.2cm}
			\begin{itemize}
				\item i componenti di un Job non devono essere dichiarati come Bean.
			\end{itemize}
			\vspace{0.2cm}
		\end{minipage}
	\hline
		Spring Data
		&
		\begin{minipage}[c]{0.7\textwidth}
			\vspace{0.2cm}
			\begin{itemize}
				\item index e type nelle annotazioni dei Model devono provenire da Bean.
			\end{itemize}
			\vspace{0.2cm}
		\end{minipage}
	\hline
		JUnit
		&
		\begin{minipage}[c]{0.7\textwidth}
			\vspace{0.2cm}
			\begin{itemize}
				\item le linee di codice devono avere una test coverage superiore all'80\%.
			\end{itemize}
			\vspace{0.2cm}
		\end{minipage}
	\hline
	\caption[Lista di controllo per il software]{Lista di controllo per il software}
	\end{longtable}