%%%%%%%%%%%%%%
%  FUNZIONI  %
%%%%%%%%%%%%%%

% Serve a dare la giusta formattazione alle parole presenti nel glossario
% il nome del comando \glossary è già usato da LaTeX
% \newcommand{\glossaryItem}[1]{\textit{#1\ped{\ped{G}}}}
% \newcommand{\glossaryItem}[1]{\textit{#1}{\small$_G$}}
\newcommand{\glossaryItem}[1]{%
    \lowercase{\ifcsname @glsterms@\detokenize{#1}\endcsname}%
        #1%
    \else%
        \lowercase{\expandafter\gdef\csname @glsterms@\detokenize{#1}\endcsname{x}}%
        \emph{#1}\ped{G}%
    \fi%
}

% Serve a dare la giusta formattazione per indicare il tipo di verbale in cui e' stata presa una decisione
% Uso: \verbalRI{data}{punto}
% RI = Riunione Interna
\newcommand{\verbalRI}[2]{\textit{RI-#1-#2}}
% RE = Riunione Esterna
\newcommand{\verbalRE}[2]{\textit{RE-#1-#2}}

% Serve a dare la giusta formattazione al codice inline
\newcommand{\code}[1]{\flextt{\texttt{#1}}}

% Serve a dare la giusta formattazione a tutte le path presenti nei documenti
\newcommand{\file}[1]{\flextt{\texttt{#1}}}

% Permette di andare a capo all'interno di una cella in una tabella
\newcommand{\multiLineCell}[2][c]{
    \renewcommand{\arraystretch}{1}
    \begin{tabular}[#1]{@{}l@{}}
      #2
    \end{tabular}
}

% Serve per creare un paragrafo il cui contenuto lasci un interlinea dal titolo del paragrafo.
% Di default il contenuto del paragrafo inizia subito alla destra del titolo del paragrafo
\newcommand{\myparagraph}[1]{\paragraph{#1}\mbox{} \mbox{}}

\newcommand{\mysubparagraph}[1]{\subparagraph{#1}\mbox{} \mbox{}}
