%**************************************************************
% Importazione package
%**************************************************************
\usepackage{graphbox}

\usepackage[margin=0.75in,top=1in,bottom=1.5in]{geometry}
%lista item nelle tabelle
\usepackage{booktabs}

% per scrivere in italiano e in inglese;
\usepackage[english, italian]{babel}
% specifica con quale codifica bisogna leggere i file
\usepackage[utf8]{inputenc}

% CONTROLLA
% imposta lo stile italiano per i paragrafi
\usepackage{parskip}

% package che modifica i caption di immagini, tabelle etc.
\usepackage{caption}

% package che importa il simbolo dell'euro
\usepackage{eurosym}

% numera i sottoparagrafi - fino a 5
\setcounter{secnumdepth}{5}

% elenca anche i sottoparagrafi nell'indice - fino a 5
\setcounter{tocdepth}{5}

% package che permette di definire dei colori
\usepackage[usenames,dvipsnames]{color}

% package che permette di usare il comando "paragraph" come subsubsubsection!
\usepackage{titlesec}

% package che permette di inserire img e table
% posizionati dove c'e comando
% \begin{figure}[H] ... \end{figure}
% evitando che venga spostato in automatico
\usepackage{float}

% package che permette insert di url e collegamenti
\usepackage[colorlinks=true]{hyperref}

% package per uso di immagini
\usepackage{graphicx}

% package che permette di avere tabelle su più pagine
\usepackage{longtable}

% package per multirow per tabelle
\usepackage{multirow}

% package per avere sfondo su celle
\usepackage[table]{xcolor}

% package per header e footer
\usepackage{fancyhdr}

% package che permette di inserire caratteri speciali
\usepackage{textcomp}

% package che permette di aggiustare i margini e centrare tabelle e figure
\usepackage{changepage}

% Per inserire del codice sorgente formattato
\usepackage{listings}

% Per modificare l'altezza delle righe delle tabelle
\usepackage{makecell}
