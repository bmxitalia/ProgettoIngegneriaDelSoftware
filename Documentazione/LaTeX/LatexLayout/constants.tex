%%%%%%%%%%%%%%
%  COSTANTI  %
%%%%%%%%%%%%%%
%\input{../../../LatexLayout/functions.tex} -> questa inclusione c'è già in global.tex

% scopi prodotti e descrizione glossario
\newcommand{\ScopoProdotto}{
    Lo scopo del \glossaryItem{prodotto} è realizzare un set di funzioni basate su \glossaryItem{Elasticsearch} e
    \glossaryItem{Kibana} per interpretare i dati raccolti da un \glossaryItem{Agent}.
    I dati interpretati forniranno a \glossaryItem{DevOps} statistiche e informazioni utili per comprendere il
    funzionamento della propria applicazione. In particolare si richiede lo sviluppo di un motore di generazione
    di \glossaryItem{metriche} da \glossaryItem{trace}, un motore di generazione di \glossaryItem{baseline}
    basato sulle metriche del punto precedente, e un motore di gestione di \glossaryItem{critical event}.
}

\newcommand{\DescrizioneGlossario}{
    All'interno del documento sono presenti termini che possono assumere significati diversi a seconda del contesto.
    Per evitare ambiguità, i significati dei termini complessi adottati nella stesura della documentazione sono
    contenuti nel documento \vGlossario{}. Per segnalare un termine del testo presente all'interno del Glossario
    verrà aggiunta una \glossaryItem{} a pedice e il testo sarà in corsivo.
}

% altre costanti
\newcommand{\GroupName}{MILCTdev}
\newcommand{\GroupEmail}{milctdev.team@gmail.com}
\newcommand{\ProjectName}{OpenAPM}
\newcommand{\Proponente}{Kirey Group}
\newcommand{\Committente}{Prof. Tullio Vardanega \\ Prof. Riccardo Cardin}
\newcommand{\Committenteinriga}{Prof. Tullio Vardanega e Prof. Riccardo Cardin}

% Nomi
\newcommand{\Mattia}{Mattia Bano}
\newcommand{\Carlo}{Carlo Munarini}
\newcommand{\Luca}{Luca Dal Medico}
\newcommand{\Isacco}{Isacco Maculan}
\newcommand{\Leonardo}{Leonardo Nodari}
\newcommand{\Tommaso}{Tommaso Carraro}
\newcommand{\Cristian}{Dragos Cristian Lizan}


% Versione documenti
\newcommand{\VersioneG}{3.0.0}
\newcommand{\VersionePQ}{4.0.0}
\newcommand{\VersioneNP}{4.0.0}
\newcommand{\VersionePP}{4.0.0}
\newcommand{\VersioneAR}{4.0.0}
\newcommand{\VersioneSF}{1.0.0}
\newcommand{\VersioneMU}{2.0.0}
\newcommand{\VersioneMS}{2.0.0}
\newcommand{\VersionePB}{2.0.0}

% Nomi documenti
\newcommand{\Glossario}{Glossario}
\newcommand{\PianoQualifica}{Piano di Qualifica}
\newcommand{\NormeProgetto}{Norme di Progetto}
\newcommand{\PianoProgetto}{Piano di Progetto}
\newcommand{\AnalisiRequisiti}{Analisi dei Requisiti}
\newcommand{\StudioFattibilita}{Studio di Fattibilità}
\newcommand{\ManualeUtente}{Manuale Utente}
\newcommand{\VerbaleInterno}{Verbale Interno}
\newcommand{\VerbaleEsterno}{Verbale Esterno}
\newcommand{\TecnologyBaseline}{Technology Baseline}
\newcommand{\ProductBaseline}{Product Baseline}
\newcommand{\ManualeSviluppatore}{Manuale Sviluppatore}

% Ruoli dei collaboratori
\newcommand{\Amministratore}{Amministratore}
\newcommand{\Responsabile}{Responsabile}
\newcommand{\Verificatore}{Verificatore}
\newcommand{\Analista}{Analista}
\newcommand{\Programmatore}{Programmatore}
\newcommand{\Progettista}{Progettista}


% Nome documento + versione
\newcommand{\vGlossario}{\textit{\Glossario \hspace{0.5mm} v\VersioneG{}}}
\newcommand{\vPianoDiQualifica}{\textit{\PianoQualifica \hspace{0.5mm} v\VersionePQ{}}}
\newcommand{\vNormeDiProgetto}{\textit{\NormeProgetto \hspace{0.5mm} v\VersioneNP{}}}
\newcommand{\vPianoDiProgetto}{\textit{\PianoProgetto \hspace{0.5mm} v\VersionePP{}}}
\newcommand{\vStudioDiFattibilita}{\textit{\StudioFattibilita \hspace{0.5mm} v\VersioneSF{}}}
\newcommand{\vAnalisiDeiRequisiti}{\textit{\AnalisiRequisiti \hspace{0.5mm} v\VersioneAR{}}}
\newcommand{\vSpecificaTecnica}{\textit{\SpecificaTecnica \hspace{0.5mm} v\VersioneST{}}}
\newcommand{\vManualeUtente}{\textit{\ManualeUtente \hspace{0.5mm} v\VersioneMU{}}}
\newcommand{\vManualeSviluppatore}{\textit{\ManualeSviluppatore \hspace{0.5mm} v\VersioneMS{}}}
\newcommand{\vProductBaseline}{\textit{\ProductBaseline \hspace{0.5mm} v\VersionePB{}}}

%Revisioni
\newcommand{\RR}{Revisione dei Requisiti}
\newcommand{\RP}{Revisione di Progetto}
\newcommand{\RQ}{Revisione di Qualifica}
\newcommand{\RA}{Revisione di Accettazione}
